\documentclass[12pt,a4paper]{article}
\usepackage{titlesec}
\usepackage{fancyhdr}
\newcommand{\sectionbreak}{\clearpage}
% !TEX encoding = UTF-8 Unicode
\usepackage[utf8]{inputenc} % set input encoding (not needed with XeLaTeX)
\usepackage{graphicx} % support the \includegraphics command and options
\usepackage{pdfpages} % use for the \includepdf command
\usepackage{pax} % use for keeping any hyperlinks when using \includepdf

\usepackage[parfill]{parskip} % begin paragraphs with an empty line rather than an indent
\usepackage{booktabs} % for much better looking tables
\usepackage{array} % for better arrays (eg matrices) in maths
\usepackage{paralist} % very flexible & customisable lists (eg. enumerate/itemize, etc.)
\usepackage{enumitem} % allows you to add styles to lists
\usepackage{verbatim} % adds environment for commenting out blocks of text & for better verbatim
\usepackage{subfig} % make it possible to include more than one captioned figure/table in a single float
\usepackage[toc,page]{appendix}
\usepackage{listings} % use for code listings
\usepackage[hidelinks]{hyperref} % for clickable links, hide the colour boxes surrounding the links
\usepackage{apacite} % needs to be loaded here or hyperref breaks things
\usepackage{cleveref} % use to reference multiple things at once, load here or it breaks

\pagestyle{fancyplain}
\fancyhf{}
\renewcommand{\headrulewidth}{0.5pt}
\renewcommand{\footrulewidth}{0.5pt}
\setlength{\headheight}{15pt}
\fancyhead[L]{Johann Leonhardt --- 40315336}
\fancyhead[R]{SOC10101 Honours Project}
\fancyfoot[L]{}
\fancyfoot[C]{\thepage}

\lstdefinestyle{code}{
    numbers=left,
    breaklines=true,
    showstringspaces=false,
}
\lstset{basicstyle=\tiny, style=code}

\graphicspath{{./img/}}

\begin{document}

\newcommand{\HRule}{\rule{\linewidth}{0.5mm}}

\begin{titlepage}
\begin{center}
\HRule{} \\[0.4cm]
    {\large \bfseries Evaluating the Factors Which Impact the Effectiveness of Just-in-Time Manufacturing\par}
\vspace{0.2cm}
\HRule{} \\[1.5cm]

\vspace{3cm}
\begin{minipage}{0.5\textwidth}
    \begin{center} \large
        Johann Leonhardt --- 40315336      
    \end{center}
\end{minipage}

\vspace{2cm}
\begin{minipage}{1\textwidth}
    \begin{center} \large
    Submitted in partial fulfilment of \\
    the requirements of Edinburgh Napier University \\
    for the Degree of \\
    BSc (Hons) Computing Science
    \end{center}
\end{minipage}

\vfill

% Bottom of the page
\begin{minipage}{1\textwidth}
    \begin{center} \large
    School of Computing
    \end{center}
\end{minipage}

\vspace{1cm}{\large \today}
\end{center}
\end{titlepage}

\pagebreak
\section*{Authorship Declaration}
\vspace{0.5cm}
\begin{flushleft}
I, Johann Leonhardt, confirm that this dissertation and the work presented in it are my own achievement.\newline

Where I have consulted the published work of others this is always clearly attributed;\newline

Where I have quoted from the work of others the source is always given. With the exception of such quotations this dissertation is entirely my own work;\newline

I have acknowledged all main sources of help;\newline

If my research follows on from previous work or is part of a larger collaborative research project I have made clear exactly what was done by others and what I have contributed myself;\newline

I have read and understand the penalties associated with Academic Misconduct.\newline

I also confirm that I have obtained informed consent from all people I have involved in the work in this dissertation following the School's ethical guidelines.\newline
\end{flushleft}

\begin{flushleft} \large
\emph{Signed:}\\ \includegraphics[width=5cm]{signature.png}
\end{flushleft}

\vspace{.5cm}

\begin{flushleft} \large
\emph{Date:}\\ \today
\end{flushleft}

\vspace{.5cm}

\begin{flushleft} \large
\emph{Matriculation no:}\\ 40315336
\end{flushleft}

\pagebreak
\section*{General Data Protection Regulation Declaration}
\vspace{0.5cm}
\begin{flushleft}
Under the General Data Protection Regulation (GDPR) (EU) 2016/679, the University cannot disclose your grade to an unauthorised person. However, other students benefit from studying dissertations that have their grades attached.\newline
\vspace{0.5cm}

Please sign your name below one of the options below to state your preference.\newline
\vspace{0.5cm}

The University may make this dissertation, with indicative grade, available to others.\newline
\includegraphics[width=5cm]{signature.png}
\vspace{2.5cm}

The University may make this dissertation available to others, but the grade may not be disclosed.\newline
\vspace{3cm}

The University may not make this dissertation available to others.\newline
\end{flushleft}

\pagebreak

\begin{abstract}
% fill the abstract in here
\end{abstract}
\pagebreak

\tableofcontents
\newpage

\listoffigures
\newpage

\renewcommand\lstlistlistingname{List of Listings}
\lstlistoflistings{}
\newpage

\listoftables
\newpage

\section*{Acknowledgements}
Insert acknowledgements here
\subsection*{}
\newpage

\section{Introduction}

This project aims to see how effective it is to write a multi-agent system in \citeA{elixir2021elixir}.

A multi-agent system consists of several intelligent agents that interact with each other to achieve either common or conflicting goals.
In complex systems, agents can be used instead of humans for decision making by giving them a set of rules to follow.
One of the areas where the use of agents seems promising is for managing supply chains.

Supply chains naturally have multiple entities that compose them, each of which is trying to ensure they can meet demand requirements and not lose money.
Current supply chains rely on long-standing orders from a select few companies, however, using agents to manage the buying and selling of goods would allow the benefits of short term contracts to be evaluated faster.

Due to the complexity of a supply chain where an agent needs to make decisions about procurement, production and selling, this seems like an ideal test case to use for benchmarking technologies for implementing multi-agent systems.
This project will use the scenario that was used in the TAC SCM competition, which was the manufacturing of PCs from individual components~\cite{sadeh2003tac}.
Using the theme of an existing competition ensures that the results of this benchmark are relevant to already existing research interests.

Elixir is a functional programming language that uses the Erlang VM\@.
This gives Elixir a lot of advantages for writing fault-tolerant distributed systems.
Such as lightweight threads which do not share state and can only communicate using messages.
Elixir could be considered a good technology to use for implementing multi-agent systems as the concepts behind agent-based design can be easily mapped to existing functionality in the language itself.

Most multi-agent systems have been implemented using a framework called JADE~\cite{bellifemine1999jade}.
This is a Java framework that allows you to create FIPA compliant agent systems.
FIPA is a set of agent system specifications that define how to identify and communicate with agents~\cite{obrien1998fipa}.
This project tries to see whether Elixir as a language is better suited for writing agents or whether the specification compliance of JADE is enough to ensure it is a good choice.

This project aims to implement a multi-agent system in Elixir and then to reimplement the same system using JADE\@.
After this is done, both of these systems can be compared to find which one is superior.

This project could be important for other computer scientists studying multi-agent systems as Elixir has not yet been used in this area.
If the Elixir implementation is more performant and easier to write than a version written in JADE, it could show that more modern technologies should be considered instead of using JADE\@.

\subsection{Deliverables}

There will be three deliverables for this project.
The first deliverable will be the complete multi-agent benchmarking system written in Elixir.
The second deliverable will be the comparison benchmarking system written in JADE\@.
The final deliverable will be the results of the two benchmark systems.

\subsection{Research Questions}

This project seeks to answer the following research questions:

\begin{enumerate}
    \item Compared to JADE, to what extent is it easier to write a multi-agent system in Elixir?
    \item Does a system written in Elixir perform better than a system written in JADE\@?
\end{enumerate}

The first question aims to see whether a system implemented in Elixir is more succinctly written than an equivalent system implemented using JADE\@.
The second question tries to see whether a system written in Elixir performs better in terms of CPU and memory usage than a system written with JADE\@.

\subsection{Project Overview}

This project will have several milestones that need to be completed.

\begin{description}[style=nextline]
    \item [Literature Review] Background research on existing multi-agent systems primarily looking at supply chains and the technology that they use will need to be done to provide a foundation for the implementation of this project.
    \item [Methodology] The multi-agent system will need to be designed so that it can be implemented with the same behaviour in both Elixir and JADE\@.
    \item [Implementation] After the system has been designed it will need to be implemented in both Elixir and JADE\@.
    \item [Results] Once the systems have been fully implemented they can be benchmarked and compared to answer the research questions.
    \item [Conclusion] The implications of the answers to the research questions will be discussed and any future work that might need to be done will be mentioned.
\end{description}


\section{Literature Review}

Introduction here.

\subsection{Modelling Supply Chains}

\subsection{Agent-based Design}

\subsection{Technologies for Agent-based Modelling}

\subsubsection{Erlang}

The Erlang\footnote{\url{https://www.erlang.org/}} programming language was introduced by \citeA{armstrong1990erlang} as an experimental new programming language to be used for programming telephony systems.
Erlang is designed around using concurrent processes with no shared memory, which communicate using asynchronous message passing.
All of this is built into the language definition.~\cite{armstrong2007history}
The fact that this behaviour is implemented on a language basis and not as a third-party library is beneficial as it allows us to implement the basic Actor principle of message passing entirely using Erlang processes.

\citeA{varela2004modelling} describes a method for implementing agents using multiple Erlang processes to represent individual components of an agent.
These processes communicate with one another to execute overall actions for the agent, and one or more processes are dedicated to communicating with other agents in the system.
They also describe the notion of a supervisor tree, which is a tree-like hierarchy of agents where each agent monitors the state of any agents below it.
This allows fault tolerance as long as the supervisor stores the initial state of all the agents it monitors.

\citeA{di2005using} evaluated Erlang on three metrics: Agent Model Compliance, Support for Rationality and Support for Distribution.
They found that while Erlang fully met the criteria for distribution support, it did not fully support agent behaviour composition and extension, which would be required for agent model compliance.
They also found that Erlang lacked logical inference support to provide rationality to agents.
They propose the eXAT (\textbf{e}rlang e\textbf{X}perimental \textbf{A}gent \textbf{T}ool) platform as a suitable library for fixing these issues.~\cite{di2003exat}
However, \citeA{peregud2012implementing} found that the eXAT platform was cumbersome to write models with, was practically discontinued since 2005 and has no community support and no production systems using it.

\citeA{santana2017interscsimulator} have used Erlang to model a large scale traffic simulation using millions of agents.
They note that the parallelism provided by Erlang processes and the ability to run Erlang code distributed on multiple machines simultaneously are the main factors for the performance they achieved.
However, they mention that the drawbacks of Erlang are a lack of process synchronisation mechanisms and a general lack of IDE tooling.

While Erlang has been used for implementing other types of scientific modelling, it has yet to be used for modelling supply chains.

\subsubsection{Elixir}

The Elixir\footnote{\url{https://elixir-lang.org/}} programming language can be considered to be a more modern version of Erlang.
Elixir compiles to the same byte code that the Erlang runtime uses.
This allows Elixir code to call code written in Erlang with no runtime overhead.
Elixir uses a syntax similar to Ruby, which is more modern than the Prolog inspired syntax of Erlang.~\cite{loder2016erlang}

As Elixir is a fairly new programming language it has yet to be used for any type of modelling research.
\citeA{fedrecheski2016elixir} compared Elixir with Java for use in Internet of Things software.
They found that while the Java implementation used 5\% to 15\% less CPU load, the Elixir version was shorter in terms of lines of code, consumed significantly less memory, and handled HTTP responses better under heavy load.

While Elixir has not yet found much use in the scientific community, it has been used for web development and real-time applications.
Discord had been an early adopter of Elixir for its messaging platform and was able to scale its application to handle five million concurrent users using it.~\cite{vishnevskiy2017discord}
\citeA{pinterest2017introducing} have used Elixir to implement a notification system.
They found that the Elixir version had ten times fewer lines of code than the original Java version, and ran faster as well.

\subsubsection{JADE}

JADE (\textbf{J}ava \textbf{A}gent \textbf{DE}velopment Framework)\footnote{\url{https://jade.tilab.com/}} is a Java framework for developing agent-based applications, and was one of the first FIPA specification compliant frameworks.~\cite{bellifemine1999jade}
JADE provides a platform for executing agents via composable behaviours and allowing both local and distributed agents to communicate with each other.
One optimisation that JADE provides is that FIPA ACL messages are transported between JADE agents on the same platform as Java objects, instead of needing to be serialised into a string format.
For agents on different platforms the message is converted as required but this provides a performance boost for local agents.

\citeA{bergenti2020first} note that in the 20 years that the JADE framework has been available, it has been extensively used in academia for software agent research as well as being used to introduce students to agent-oriented programming.
However, \shortciteA{bergenti2020first} also mention that students using JADE struggle with the complexity of the framework, if they do not have any prior knowledge of software agents.
They propose Jadescript as a language to provide a dedicated syntax for constructing and using JADE agents.

JADE has been used by \citeA{podobnik2006crocodileagent} for implementing an agent to compete in TAC SCM\@.
As they were using an IKB agent model, they mention that a benefit of JADE was that they could separate each layer onto a different computer.
While this results in a more complex system due to requiring lots of intercommunication, they still managed to reach the semi-finals of the competition once and the quarter-finals twice.~\cite{collins2009flexible}

\subsubsection{NetLogo}

NetLogo\footnote{\url{https://ccl.northwestern.edu/netlogo/}} is a multi-agent programming language which provides an environment for modelling complex systems over time
It has been designed for use in both education and research, and since it has been written using Java it can be run on any major operating system.
NetLogo uses agents called `turtles' to move over and interact with a grid of `patches'.
Both of these can be programmed to have certain behaviours.
NetLogo uses a primarily graphical interface to make it easy to view and modify the simulation while it is running.~\cite{tisue2004netlogo}

While NetLogo is popular in general for agent-based modelling, only a few SCM models have used it.
\citeA{arvitrida2015competition} used NetLogo to model the effect of competition and collaboration on supply chain performance.
They selected NetLogo for being relatively simple but still providing all the features that they needed for this model.

\subsubsection{SARL}

SARL\footnote{\url{http://www.sarl.io/}} is a general-purpose agent-based programming language that aims to be both platform and architecturally agnostic.
SARL provides native support for agent-oriented first-class abstractions but does not force the programmer to use them in any specific way.
The reason why SARL is designed to be agnostic is the belief that a more general and less research focused or theoretical approach is required for agent-oriented programming to make a more significant impact on mainstream software engineering.
SARL compiles to Java byte code and can be run on the Janus platform, although other platforms can be used as well.~\cite{rodriguez2014sarl}

As SARL is a relatively new agent-oriented programming language, it has yet to be used for SCM and has also not seen much use for general agent-based modelling yet.

\subsubsection{Comparison Between Technologies}

This section compares all the previously mentioned technologies, in order to evaluate them for this project.
A short example program has written for each technology, so that the ease of writing a multi-agent system can be objectively compared.
See \Cref{tab:tech_comparison} for an overview of the comparison.

The example multi-agent model that was written for each technology is a `counter agent' program.
This program has two agents, the first agent discovers the second agent and then counts to three before telling the other agent to terminate.
This is a very simple example of multiple agents interacting with each other, and is designed to show how much support a given technology has for agent-oriented primitives such as message passing and defining repeating actions.
The lines of code for these programs has been counted as the number of lines not blank or containing a comment.

\Cref{lst:erl_app,lst:erl_sup,lst:erl_counter,lst:erl_partner} shows the counter agent example written in Erlang.
This example uses the Open Telecom Platform (OTP) generic server behaviour to implement message receiving and agent behaviour.
This allows the agents to be managed by an OTP process supervisor, as seen in \cref{lst:erl_sup}.
The benefit of this is that agents that crash can then be automatically restarted by the supervisor.
This example program comes to 81 lines of code.

eXAT has been compared with Java and JADE by \citeA{di2005using}.
They mention that since Java does not support function clause pattern matching and requires the use of objects to provide a symbolic language, eXAT is superior for implementing and agent-based model.
While \shortciteA{peregud2012implementing} is overall critical of eXAT, they mention a benefit of it is that it allows a large amount of eXAT agents to run at once regardless of size.
Compared to JADE which suggests the use of fewer but larger agents in general.
It should be noted that both of these positives are a result of the Erlang language and not something entirely specific to the eXAT platform.

It was not possible to write a counter agent example in eXAT\@.
The platform depends on deprecated dependencies causing it to fail to compile on \verb|Erlang/OTP 23.2.6|.
As a result of this and the fact that the last update to eXAT was done by \shortciteA{peregud2012implementing} in 2012, this project seems to be completely inactive.

\Cref{lst:ex_app,lst:ex_counter,lst:ex_partner} shows the counter agent example in Elixir.
This program is implemented in 71 lines of code.
Comparing this to the Erlang program we can see that Elixir allows us to write code that is almost semantically equivalent to Erlang, just using a more modern syntax.
We can see the Erlang compatibility of Elixir by the call to \verb|:timer.send_interval/2| on line 25 of \cref{lst:ex_counter}.
This is calling a function in the \verb|timer| module of the Erlang standard library to send a message every second.
One of the reasons why the Elixir code is shorter than the Erlang code is due to Elixir's use of macros for metaprogramming.
The Elixir statement \verb|use GenServer| is a macro which provides default implementations for functions such as \verb|handle_call/3| and \verb|handle_cast/2| which need to be implemented in Erlang manually (see lines 43--47 in \cref{lst:erl_counter}).

\citeA{christos2016agent} have compared the experience writing a supply chain model in both NetLogo and JADE\@.
They found that NetLogo not requiring an external IDE for programming and having easily embeddable graphical controls made it easy to use.
However, they conclude that JADE programs provide more flexible and robust programs assuming the user is willing to spend the time to learn the technology.

\Cref{lst:netlogo} shows the counter agent example implemented in NetLogo.
This is the shortest implementation of the counter agent at only 32 lines of code.
This conciseness however comes at a price.
NetLogo does not have a native way to send messages between `turtles' and has no way for a `turtle' to refuse to do an action it has been asked to do.
While this makes example programs short and easy to read, it would likely cause issues as models grow larger and more complex.
NetLogo does have a library\footnote{\url{https://users.uom.gr/~iliass/projects/NetLogo/}} to support sending messages between agents, however, this library only supports \verb|NetLogo 5| whereas the current version is \verb|NetLogo 6.2|.~\cite{sakellariou2008enhancing}

For this project I have chosen to use Elixir for implementing my supply chain model.
I believe that the benefits of using a language that can leverage the features of Erlang while simultaneously having a more modern syntax, outweighs the potential issues of using a language that has not yet been used for research purposes.

\begin{table}[h]
    \centering
    \caption{Technology Comparison}\label{tab:tech_comparison}
    \begin{tabular}{@{}lllll@{}}
        \toprule
        Name & SCM & FIPA & Difficulty & Active Project\\
        \midrule
        Erlang & No & No &  & Yes\\
        eXAT & No & Yes & N/A & No\\
        Elixir & No & No &  & Yes\\
        JADE & Yes & Yes &  & No?\\
        NetLogo & Yes & Plugin &  & Yes\\
        SARL & No & No &  & Yes\\
        \bottomrule
    \end{tabular}
\end{table}


\section{Methodology}

This section covers the design of the system as well as how it will be benchmarked.

\subsection{Implementation}

This section will introduce some of the common details about both of the implementations.
Such as how the agent will be designed in general and how each agent will communicate with each other.

Docker will be used to test the distributed system.
This is because it provides a fast way of starting multiple instances at once.

\subsubsection{Elixir}

Each of the parts of the agent will be using an OTP supervisor to manage the processes for that part.
Each agent will share the majority of the code to avoid duplicated code.
Then at start time, the correct behaviour for each agent will be enabled according to configurations files or environment variables.

Each individual agent will run on a separate node which will be\\auto-discovered using the `libcluster' library.

\subsubsection{JADE}

Not yet sure how to implement the JADE agents.

The same idea of enabling behaviours depending on agent type should still be possible.

\subsection{Benchmarking}

The issue with using Docker is that it makes benchmarking more difficult.
Running the entire system virtualised on one machine is not representative of a real-world deployment.

An alternative approach would be to run all the containers in a cloud provider such as Azure.
From some quick experimentation, there does not seem to be a way to measure the resource usage of a single container in Azure, just the whole container group.

Running the system on individual machines is also an option, either rented in the cloud or physical machines such as Raspberry Pis, but this would likely result in a lot of time required to set these machines up.

\section{Results}

Look at the performance of both systems with different configurations.
For example 1 customer and 2 raw producers.
Then add new customer agents and new producer agents to see how the trading agent handles it.
The final benchmark would be a setup like that for TAC SCM, which would be 1 customer and 2 agents each for 4 unique components.

\section{Analysis}

It is expected from \citeA{fedrecheski2016elixir} that Elixir will perform slightly worse in terms of CPU usage due to the overhead of scheduling potentially hundreds of processes, but that it should perform better in terms of memory usage due to requiring less heavy objects than Java.
It is also expected that Elixir will use fewer lines of code compared with JADE\@.

However, JADE requires a Java version of 1.8, meaning many performance improvements will not be available.
\citeA{fedrecheski2016elixir} compares Java 1.8 with a fairly old version of Elixir and Erlang (1.2 and OTP 18.2).
The current versions are Elixir 1.11 and Erlang OTP 23.3 which may produce more performant code.

\section{Conclusion}

Answer the two research questions using the analysis from the previous section.

Draw a general conclusion from this and list any future work that could be done.

\setlength{\bibitemsep}{0.5\baselineskip}
\interlinepenalty=10000 % Disallows page breaks from occurring in a line

\bibliographystyle{apacite}
\bibliography{bibliography}

\pagebreak
\begin{appendices}
\section{Project Overview}

\includepdf[pages=-, pagecommand={\thispagestyle{empty}}]{../management/ipo.pdf}

\section{Second Formal Review Output}
Insert a copy of the project review form you were given at the end of the review by the second marker

\section{Diary Sheets}

\includepdf[pages=-, pagecommand={\thispagestyle{plain}}]{../management/project_diary/week1.pdf}
\includepdf[pages=-, pagecommand={\thispagestyle{plain}}]{../management/project_diary/week2.pdf}
\includepdf[pages=-, pagecommand={\thispagestyle{plain}}]{../management/project_diary/week3.pdf}
\includepdf[pages=-, pagecommand={\thispagestyle{plain}}]{../management/project_diary/week4.pdf}

\end{appendices}


\end{document}
