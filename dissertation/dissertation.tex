\documentclass[12pt,a4paper]{article}
\usepackage{titlesec}
\usepackage{fancyhdr}
\newcommand{\sectionbreak}{\clearpage}
% !TEX encoding = UTF-8 Unicode
\usepackage[utf8]{inputenc} % set input encoding (not needed with XeLaTeX)
\usepackage{graphicx} % support the \includegraphics command and options
\usepackage{pdfpages} % use for the \includepdf command
\usepackage{pax} % use for keeping any hyperlinks when using \includepdf

\usepackage[parfill]{parskip} % begin paragraphs with an empty line rather than an indent
\usepackage{booktabs} % for much better looking tables
\usepackage{array} % for better arrays (eg matrices) in maths
\usepackage{paralist} % very flexible & customisable lists (eg. enumerate/itemize, etc.)
\usepackage{enumitem} % allows you to add styles to lists
\usepackage{verbatim} % adds environment for commenting out blocks of text & for better verbatim
\usepackage{subfig} % make it possible to include more than one captioned figure/table in a single float
\usepackage[toc,page]{appendix}
\usepackage{listings} % use for code listings
\usepackage[hidelinks]{hyperref} % for clickable links, hide the colour boxes surrounding the links
\usepackage{apacite} % needs to be loaded here or hyperref breaks things
\usepackage{cleveref} % use to reference multiple things at once, load here or it breaks

\pagestyle{fancyplain}
\fancyhf{}
\renewcommand{\headrulewidth}{0.5pt}
\renewcommand{\footrulewidth}{0.5pt}
\setlength{\headheight}{15pt}
\fancyhead[L]{Johann Leonhardt --- 40315336}
\fancyhead[R]{SOC10101 Honours Project}
\fancyfoot[L]{}
\fancyfoot[C]{\thepage}

\lstdefinestyle{code}{
    numbers=left,
    breaklines=true,
    showstringspaces=false,
    captionpos=b,
}
\lstset{basicstyle=\tiny, style=code}

\graphicspath{{./img/}}

\begin{document}

\newcommand{\HRule}{\rule{\linewidth}{0.5mm}}

\begin{titlepage}
\begin{center}
\HRule{} \\[0.4cm]
    {\large \bfseries Evaluating the Factors Which Impact the Effectiveness of Just-in-Time Manufacturing\par}
\vspace{0.2cm}
\HRule{} \\[1.5cm]

\vspace{3cm}
\begin{minipage}{0.5\textwidth}
    \begin{center} \large
        Johann Leonhardt --- 40315336      
    \end{center}
\end{minipage}

\vspace{2cm}
\begin{minipage}{1\textwidth}
    \begin{center} \large
    Submitted in partial fulfilment of \\
    the requirements of Edinburgh Napier University \\
    for the Degree of \\
    BSc (Hons) Computing Science
    \end{center}
\end{minipage}

\vfill

% Bottom of the page
\begin{minipage}{1\textwidth}
    \begin{center} \large
    School of Computing
    \end{center}
\end{minipage}

\vspace{1cm}{\large \today}
\end{center}
\end{titlepage}

\pagebreak
\section*{Authorship Declaration}
\vspace{0.5cm}
\begin{flushleft}
I, Johann Leonhardt, confirm that this dissertation and the work presented in it are my own achievement.\newline

Where I have consulted the published work of others this is always clearly attributed;\newline

Where I have quoted from the work of others the source is always given. With the exception of such quotations this dissertation is entirely my own work;\newline

I have acknowledged all main sources of help;\newline

If my research follows on from previous work or is part of a larger collaborative research project I have made clear exactly what was done by others and what I have contributed myself;\newline

I have read and understand the penalties associated with Academic Misconduct.\newline

I also confirm that I have obtained informed consent from all people I have involved in the work in this dissertation following the School's ethical guidelines.\newline
\end{flushleft}

\begin{flushleft} \large
\emph{Signed:}\\ \includegraphics[width=5cm]{signature.png}
\end{flushleft}

\vspace{.5cm}

\begin{flushleft} \large
\emph{Date:}\\ \today
\end{flushleft}

\vspace{.5cm}

\begin{flushleft} \large
\emph{Matriculation no:}\\ 40315336
\end{flushleft}

\pagebreak
\section*{General Data Protection Regulation Declaration}
\vspace{0.5cm}
\begin{flushleft}
Under the General Data Protection Regulation (GDPR) (EU) 2016/679, the University cannot disclose your grade to an unauthorised person. However, other students benefit from studying dissertations that have their grades attached.\newline
\vspace{0.5cm}

Please sign your name below one of the options below to state your preference.\newline
\vspace{0.5cm}

The University may make this dissertation, with indicative grade, available to others.\newline
\includegraphics[width=5cm]{signature.png}
\vspace{2.5cm}

The University may make this dissertation available to others, but the grade may not be disclosed.\newline
\vspace{3cm}

The University may not make this dissertation available to others.\newline
\end{flushleft}

\pagebreak

\begin{abstract}

A multi-agent system is a computer system where multiple intelligent agents interact with each other using messages to solve a common goal.
Most multi-agent systems have been developed using a Java framework called JADE\@.
This dissertation evaluates whether the programming language Elixir, which has a lot of features that can easily be used for agent-based design, performs better than JADE at implementing a multi-agent system.
The theme of using a supply chain system was chosen due to the popularity of the TAC SCM competition.
This dissertation measures the lines of source code required to implement a system in Elixir against the amount required for JADE\@.
It also measures the CPU utilisation, memory usage and the time taken to complete an experiment for both systems.
After completing five experiments this dissertation finds that Elixir performs better on all metrics apart from CPU utilisation.
This dissertation concludes that Elixir is a viable choice for implementing multi-agent systems.

\end{abstract}
\pagebreak

\tableofcontents
\newpage

\listoffigures
\newpage

\renewcommand\lstlistlistingname{List of Listings}
\lstlistoflistings{}
\newpage

\listoftables
\newpage

\section*{Acknowledgements}

Many thanks to my supervisor, Dr Simon T. Powers, for helping me take my vague initial idea and actualise this dissertation.

Dedicated to the work of Joe Armstrong and Fabio Bellifemine.

\newpage

\section{Introduction}

This project aims to see how effective it is to write a multi-agent system in \citeA{elixir2021elixir}.

A multi-agent system consists of several intelligent agents that interact with each other to achieve either common or conflicting goals.
In complex systems, agents can be used instead of humans for decision making by giving them a set of rules to follow.
One of the areas where the use of agents seems promising is for managing supply chains.

Supply chains naturally have multiple entities that compose them, each of which is trying to ensure they can meet demand requirements and not lose money.
Current supply chains rely on long-standing orders from a select few companies, however, using agents to manage the buying and selling of goods would allow the benefits of short term contracts to be evaluated faster.

Due to the complexity of a supply chain where an agent needs to make decisions about procurement, production and selling, this seems like an ideal test case to use for benchmarking technologies for implementing multi-agent systems.
This project will use the scenario that was used in the TAC SCM competition, which was the manufacturing of PCs from individual components~\cite{sadeh2003tac}.
Using the theme of an existing competition ensures that the results of this benchmark are relevant to already existing research interests.

Elixir is a functional programming language that uses the Erlang VM\@.
This gives Elixir a lot of advantages for writing fault-tolerant distributed systems.
Such as lightweight threads which do not share state and can only communicate using messages.
Elixir could be considered a good technology to use for implementing multi-agent systems as the concepts behind agent-based design can be easily mapped to existing functionality in the language itself.

Most multi-agent systems have been implemented using a framework called JADE~\cite{bellifemine1999jade}.
This is a Java framework that allows you to create FIPA compliant agent systems.
FIPA is a set of agent system specifications that define how to identify and communicate with agents~\cite{obrien1998fipa}.
This project tries to see whether Elixir as a language is better suited for writing agents or whether the specification compliance of JADE is enough to ensure it is a good choice.

This project aims to implement a multi-agent system in Elixir and then to reimplement the same system using JADE\@.
After this is done, both of these systems can be compared to find which one is superior.

This project could be important for other computer scientists studying multi-agent systems as Elixir has not yet been used in this area.
If the Elixir implementation is more performant and easier to write than a version written in JADE, it could show that more modern technologies should be considered instead of using JADE\@.

\subsection{Deliverables}

There will be three deliverables for this project.
The first deliverable will be the complete multi-agent benchmarking system written in Elixir.
The second deliverable will be the comparison benchmarking system written in JADE\@.
The final deliverable will be the results of the two benchmark systems.

\subsection{Research Questions}

This project seeks to answer the following research questions:

\begin{enumerate}
    \item Compared to JADE, to what extent is it easier to write a multi-agent system in Elixir?
    \item Does a system written in Elixir perform better than a system written in JADE\@?
\end{enumerate}

The first question aims to see whether a system implemented in Elixir is more succinctly written than an equivalent system implemented using JADE\@.
The second question tries to see whether a system written in Elixir performs better in terms of CPU and memory usage than a system written with JADE\@.

\subsection{Project Overview}

This project will have several milestones that need to be completed.

\begin{description}[style=nextline]
    \item [Literature Review] Background research on existing multi-agent systems primarily looking at supply chains and the technology that they use will need to be done to provide a foundation for the implementation of this project.
    \item [Methodology] The multi-agent system will need to be designed so that it can be implemented with the same behaviour in both Elixir and JADE\@.
    \item [Implementation] After the system has been designed it will need to be implemented in both Elixir and JADE\@.
    \item [Results] Once the systems have been fully implemented they can be benchmarked and compared to answer the research questions.
    \item [Conclusion] The implications of the answers to the research questions will be discussed and any future work that might need to be done will be mentioned.
\end{description}


\section{Literature Review}

Introduction here.

\subsection{Modelling Supply Chains}

\subsection{Agent-based Design}

\subsection{Technologies for Agent-based Modelling}

\subsubsection{Erlang}

The Erlang\footnote{\url{https://www.erlang.org/}} programming language was introduced by \citeA{armstrong1990erlang} as an experimental new programming language to be used for programming telephony systems.
Erlang is designed around using concurrent processes with no shared memory, which communicate using asynchronous message passing.
All of this is built into the language definition.~\cite{armstrong2007history}
The fact that this behaviour is implemented on a language basis and not as a third-party library is beneficial as it allows us to implement the basic Actor principle of message passing entirely using Erlang processes.

\citeA{varela2004modelling} describes a method for implementing agents using multiple Erlang processes to represent individual components of an agent.
These processes communicate with one another to execute overall actions for the agent, and one or more processes are dedicated to communicating with other agents in the system.
They also describe the notion of a supervisor tree, which is a tree-like hierarchy of agents where each agent monitors the state of any agents below it.
This allows fault tolerance as long as the supervisor stores the initial state of all the agents it monitors.

\citeA{di2005using} evaluated Erlang on three metrics: Agent Model Compliance, Support for Rationality and Support for Distribution.
They found that while Erlang fully met the criteria for distribution support, it did not fully support agent behaviour composition and extension, which would be required for agent model compliance.
They also found that Erlang lacked logical inference support to provide rationality to agents.
They propose the eXAT (\textbf{e}rlang e\textbf{X}perimental \textbf{A}gent \textbf{T}ool) platform as a suitable library for fixing these issues.~\cite{di2003exat}
However, \citeA{peregud2012implementing} found that the eXAT platform was cumbersome to write models with, was practically discontinued since 2005 and has no community support and no production systems using it.

\citeA{santana2017interscsimulator} have used Erlang to model a large scale traffic simulation using millions of agents.
They note that the parallelism provided by Erlang processes and the ability to run Erlang code distributed on multiple machines simultaneously are the main factors for the performance they achieved.
However, they mention that the drawbacks of Erlang are a lack of process synchronisation mechanisms and a general lack of IDE tooling.

While Erlang has been used for implementing other types of scientific modelling, it has yet to be used for modelling supply chains.

\subsubsection{Elixir}

The Elixir\footnote{\url{https://elixir-lang.org/}} programming language can be considered to be a more modern version of Erlang.
Elixir compiles to the same byte code that the Erlang runtime uses.
This allows Elixir code to call code written in Erlang with no runtime overhead.
Elixir uses a syntax similar to Ruby, which is more modern than the Prolog inspired syntax of Erlang.~\cite{loder2016erlang}

As Elixir is a fairly new programming language it has yet to be used for any type of modelling research.
\citeA{fedrecheski2016elixir} compared Elixir with Java for use in Internet of Things software.
They found that while the Java implementation used 5\% to 15\% less CPU load, the Elixir version was shorter in terms of lines of code, consumed significantly less memory, and handled HTTP responses better under heavy load.

While Elixir has not yet found much use in the scientific community, it has been used for web development and real-time applications.
Discord had been an early adopter of Elixir for its messaging platform and was able to scale its application to handle five million concurrent users using it.~\cite{vishnevskiy2017discord}
\citeA{pinterest2017introducing} have used Elixir to implement a notification system.
They found that the Elixir version had ten times fewer lines of code than the original Java version, and ran faster as well.

\subsubsection{JADE}

JADE (\textbf{J}ava \textbf{A}gent \textbf{DE}velopment Framework)\footnote{\url{https://jade.tilab.com/}} is a Java framework for developing agent-based applications, and was one of the first FIPA specification compliant frameworks.~\cite{bellifemine1999jade}
JADE provides a platform for executing agents via composable behaviours and allowing both local and distributed agents to communicate with each other.
One optimisation that JADE provides is that FIPA ACL messages are transported between JADE agents on the same platform as Java objects, instead of needing to be serialised into a string format.
For agents on different platforms the message is converted as required but this provides a performance boost for local agents.

\citeA{bergenti2020first} note that in the 20 years that the JADE framework has been available, it has been extensively used in academia for software agent research as well as being used to introduce students to agent-oriented programming.
However, \shortciteA{bergenti2020first} also mention that students using JADE struggle with the complexity of the framework, if they do not have any prior knowledge of software agents.
They propose Jadescript as a language to provide a dedicated syntax for constructing and using JADE agents.

JADE has been used by \citeA{podobnik2006crocodileagent} for implementing an agent to compete in TAC SCM\@.
As they were using an IKB agent model, they mention that a benefit of JADE was that they could separate each layer onto a different computer.
While this results in a more complex system due to requiring lots of intercommunication, they still managed to reach the semi-finals of the competition once and the quarter-finals twice.~\cite{collins2009flexible}

\subsubsection{NetLogo}

NetLogo\footnote{\url{https://ccl.northwestern.edu/netlogo/}} is a multi-agent programming language which provides an environment for modelling complex systems over time
It has been designed for use in both education and research, and since it has been written using Java it can be run on any major operating system.
NetLogo uses agents called `turtles' to move over and interact with a grid of `patches'.
Both of these can be programmed to have certain behaviours.
NetLogo uses a primarily graphical interface to make it easy to view and modify the simulation while it is running.~\cite{tisue2004netlogo}

While NetLogo is popular in general for agent-based modelling, only a few SCM models have used it.
\citeA{arvitrida2015competition} used NetLogo to model the effect of competition and collaboration on supply chain performance.
They selected NetLogo for being relatively simple but still providing all the features that they needed for this model.

\subsubsection{SARL}

SARL\footnote{\url{http://www.sarl.io/}} is a general-purpose agent-based programming language that aims to be both platform and architecturally agnostic.
SARL provides native support for agent-oriented first-class abstractions but does not force the programmer to use them in any specific way.
The reason why SARL is designed to be agnostic is the belief that a more general and less research focused or theoretical approach is required for agent-oriented programming to make a more significant impact on mainstream software engineering.
SARL compiles to Java byte code and can be run on the Janus platform, although other platforms can be used as well.~\cite{rodriguez2014sarl}

As SARL is a relatively new agent-oriented programming language, it has yet to be used for SCM and has also not seen much use for general agent-based modelling yet.

\subsubsection{Comparison Between Technologies}

This section compares all the previously mentioned technologies, in order to evaluate them for this project.
A short example program has written for each technology, so that the ease of writing a multi-agent system can be objectively compared.
See \Cref{tab:tech_comparison} for an overview of the comparison.

The example multi-agent model that was written for each technology is a `counter agent' program.
This program has two agents, the first agent discovers the second agent and then counts to three before telling the other agent to terminate.
This is a very simple example of multiple agents interacting with each other, and is designed to show how much support a given technology has for agent-oriented primitives such as message passing and defining repeating actions.
The lines of code for these programs has been counted as the number of lines not blank or containing a comment.

\Cref{lst:erl_app,lst:erl_sup,lst:erl_counter,lst:erl_partner} shows the counter agent example written in Erlang.
This example uses the Open Telecom Platform (OTP) generic server behaviour to implement message receiving and agent behaviour.
This allows the agents to be managed by an OTP process supervisor, as seen in \cref{lst:erl_sup}.
The benefit of this is that agents that crash can then be automatically restarted by the supervisor.
This example program comes to 81 lines of code.

eXAT has been compared with Java and JADE by \citeA{di2005using}.
They mention that since Java does not support function clause pattern matching and requires the use of objects to provide a symbolic language, eXAT is superior for implementing and agent-based model.
While \shortciteA{peregud2012implementing} is overall critical of eXAT, they mention a benefit of it is that it allows a large amount of eXAT agents to run at once regardless of size.
Compared to JADE which suggests the use of fewer but larger agents in general.
It should be noted that both of these positives are a result of the Erlang language and not something entirely specific to the eXAT platform.

It was not possible to write a counter agent example in eXAT\@.
The platform depends on deprecated dependencies causing it to fail to compile on \verb|Erlang/OTP 23.2.6|.
As a result of this and the fact that the last update to eXAT was done by \shortciteA{peregud2012implementing} in 2012, this project seems to be completely inactive.

\Cref{lst:ex_app,lst:ex_counter,lst:ex_partner} shows the counter agent example in Elixir.
This program is implemented in 71 lines of code.
Comparing this to the Erlang program we can see that Elixir allows us to write code that is almost semantically equivalent to Erlang, just using a more modern syntax.
We can see the Erlang compatibility of Elixir by the call to \verb|:timer.send_interval/2| on line 25 of \cref{lst:ex_counter}.
This is calling a function in the \verb|timer| module of the Erlang standard library to send a message every second.
One of the reasons why the Elixir code is shorter than the Erlang code is due to Elixir's use of macros for metaprogramming.
The Elixir statement \verb|use GenServer| is a macro which provides default implementations for functions such as \verb|handle_call/3| and \verb|handle_cast/2| which need to be implemented in Erlang manually (see lines 43--47 in \cref{lst:erl_counter}).

\citeA{christos2016agent} have compared the experience writing a supply chain model in both NetLogo and JADE\@.
They found that NetLogo not requiring an external IDE for programming and having easily embeddable graphical controls made it easy to use.
However, they conclude that JADE programs provide more flexible and robust programs assuming the user is willing to spend the time to learn the technology.

\Cref{lst:netlogo} shows the counter agent example implemented in NetLogo.
This is the shortest implementation of the counter agent at only 32 lines of code.
This conciseness however comes at a price.
NetLogo does not have a native way to send messages between `turtles' and has no way for a `turtle' to refuse to do an action it has been asked to do.
While this makes example programs short and easy to read, it would likely cause issues as models grow larger and more complex.
NetLogo does have a library\footnote{\url{https://users.uom.gr/~iliass/projects/NetLogo/}} to support sending messages between agents, however, this library only supports \verb|NetLogo 5| whereas the current version is \verb|NetLogo 6.2|.~\cite{sakellariou2008enhancing}

For this project I have chosen to use Elixir for implementing my supply chain model.
I believe that the benefits of using a language that can leverage the features of Erlang while simultaneously having a more modern syntax, outweighs the potential issues of using a language that has not yet been used for research purposes.

\begin{table}[h]
    \centering
    \caption{Technology Comparison}\label{tab:tech_comparison}
    \begin{tabular}{@{}lllll@{}}
        \toprule
        Name & SCM & FIPA & Difficulty & Active Project\\
        \midrule
        Erlang & No & No &  & Yes\\
        eXAT & No & Yes & N/A & No\\
        Elixir & No & No &  & Yes\\
        JADE & Yes & Yes &  & No?\\
        NetLogo & Yes & Plugin &  & Yes\\
        SARL & No & No &  & Yes\\
        \bottomrule
    \end{tabular}
\end{table}


\section{Methodology}

Introduction to the chapter.

Write about what will be covered.

\subsection{System Design}

Write what the system does each round of the simulation.

Write about each type of agent.

Add message sequence diagram for a typical round of the simulation.

\subsection{Agent Design}

Describe each of the layers of the agent and what they do.

Add message sequence diagram for communication between layers.

\subsection{Elixir}

Introduction to Elixir system.

This system makes heavy use of the standard Erlang OTP behaviour\\ \verb|gen_server| (henceforth referred to as GenServer, following standard Elixir naming conventions).
A GenServer is implemented as a Client-Server relationship where a public API is used to interact with user-defined callback functions.
The core behaviour of a GenServer is to wait for a message to be received, then pass that message to the appropriate callback function before looping back to wait for the next message.
This behaviour fits neatly in line with the general behaviour of agents, where the reaction to a message can be implemented as callback functions.

The split Client-Server relationship of a GenServer can be used to define a common interface that all server implementations need to adhere to.
By defining all client-facing methods in a separate module and implementing separate callback functions in different modules, we can guarantee that all particular implementations of an agent can be interacted with in the same manner.
Callback functions can also be implemented without requiring a public interface method.
This allows the GenServer to handle arbitrary messages that have been sent to it.
This functionality has been predominantly used in this system as many messages are being sent between different layers and different network nodes.

While Elixir allows the sending of any data type between different processes, it lacks a defined message type such as the ACLMessage class in JADE\@.
For this system, a message type has been defined by using the struct feature of Elixir.
This means we can define a struct that is similar to a map where only certain keys are allowed.
Certain keys can also be enforced meaning that attempting to initialise a struct with these keys missing would be an error.
The message struct that was defined for this system is based on a subset of fields that are used in FIPA ACL messages.
The message struct contains keys for the performative, message sender, message receiver, whom to reply to, the content of the message and the message ID\@.
Helper functions for replying to, forwarding and sending the message structs have also been implemented.
Not all the messages in this system use this message format.
Where appropriate, such as the callback functions for GenServers use simpler messages such as tuples of primitives.

Instead of creating a separate project for each agent type, a single project is used for all agents.
This is beneficial as all the agent types share a common information layer implementation and reuse common knowledge layer responses.
The type of the agent is set at the start up of the application by reading the system environment variables and checking the value of the \verb|AGENT_TYPE| variable.
This variable indicates which knowledge and behaviour layer implementations should be started.
If this variable is not set or is set to an unrecognised value, then the agent application will exit with an error.

The config for each agent type is loaded in as an Elixir script file, using the Config module in the Elixir standard library.
An Elixir script file is a source code file that is run by being interpreted at runtime instead of being compiled to bytecode.
By importing the Config module, macros can be used to define key-value pairs in a scoped namespace.
This is used to store the initial knowledge for each agent type.
Since this is just an Elixir source code file, any Elixir datatypes can be used.
This means that complex types such as maps or lists of values can be used without having to parse them from a markup language like XML\@.
This is useful as it unifies the language that is used for configuration with the language that is used for code.

\subsubsection{Information Layer}

The Information layer in the Elixir system is used to handle the discovery and connection between all the different nodes in the distributed system.
The library \verb|libcluster|\footnote{\url{https://github.com/bitwalker/libcluster}} is used to handle the automatic discovery of any nodes in the local network.
This library is started in the main application supervisor tree alongside the individual supervisor trees for each layer.
It is configured to use the \verb|Gossip| clustering strategy.
This strategy uses multicast UDP (User Datagram Protocol) to send information packets across the network and forms connections with any nodes that are listening for it.

The Information layer uses the Erlang module \verb|net_kernel| function\\ \verb|monitor_nodes/1| to received messages when a new connection is formed between nodes.
This message takes the form \verb|{:nodeup, node}| where node is the name of the node that was connected to.
Once the Information layer has received this message that a node has connected, it can use the name of the node in order to request information about it.

After the Information layer has requested information about a new node, it adds it to its information filter.
First it looks up if the node is of a type that should be ignored.
If it is it sets a boolean value to be true which indicates that messages from that node should always be ignored.

The information filter is stored as a table in Erlang Term Storage.
Erlang Term Storage (ETS) is a runtime provided, in-memory key-value store.
Each process can create new tables in order to store data.
ETS allows for any type to be stored directly, meaning there is no need to convert between representations.
ETS provides constant time access to data and tables can be configured to be accessible globally.
This makes it ideal for storing information that is required by all of the layers in an agent.

When a message arrives the Information layer checks whether the sender is in the information filter or not.
If the sender is not in the filter, then the message is passed on to the Knowledge layer.
Otherwise, the message is ignored.

\subsubsection{Knowledge Layer}

The Knowledge layer is responsible for loading the initial knowledge into the agent.
It reads a config file which stores all the knowledge that an agent needs to know before a simulation starts.
Then these values are inserted into a table that is stored in ETS\@.
This allows the Behaviour layer to read and update the knowledge base as well.

When the Knowledge layer receives a forwarded message from the Information layer, it gets inserted into an ETS table called the Inbox.
This allows messages to be handled in bulk by the Behaviour layer.

At the end of the simulation the Knowledge layer prints out the total amount of money that an agent has made.

Each agent uses a separate Knowledge layer implementation in order to be able to handle unique message types.
As Elixir lacks a form of class inheritance, macros are used to ensure duplicate behaviour only needs to be written once.

\subsubsection{Behaviour Layer}

The proactive and reactive behaviour of the agent.

Is implemented as a finite state machine using the library ``GenStateMachine'' which is a wrapper class for the Erlang module \verb|gen_statem|.

Explain the behaviour of the different agent types.


\subsection{JADE}

Introduction to JADE system.

Using Java 8 as this is the last long term support version where JADE can be compiled, although due to the nature of Java bytecode a later version could be used.

Each layer is a separate agent instance.

At start-up the specific layers are started depending on the system environment.

\subsubsection{Information Layer}

Passes messages to Knowledge layer.

Is implemented in a similar style to a GenServer, this is done by extending the Cyclic Behaviour from JADE and implementing that in the class.

\subsubsection{Knowledge Layer}

Acts as the knowledge base in the system.

There is an abstract Knowledge agent which the others extend.
This agent defines common behaviour like the Elixir macros.

This class loads config values from a properties file.

Write about attempt to form agent federation and why this did not work.

Write about de-serialising agent information from XML file and why JADE custom collections made serialising JADE objects more difficult.

\subsubsection{Behaviour Layer}

Implemented using JADE's finite state machine behaviour.

Explain the behaviour of the different agent types.

\subsection{Docker}

Explain the use of Docker during development.

Explain how Docker could be used for benchmarking.

\subsection{Benchmarking}

Explain the different metrics that will be used for the experiments.

The first metric that will be measured for this benchmark is the total number of source lines of code.
This will be used to see whether an agent system implemented in Elixir is more concise than a system implemented using JADE\@.
If Elixir is indeed more concise than JADE it would throw into question whether using a framework for implementing multi-agent systems is worth it.
Having a more concise project is beneficial as the less lines of code there are, the more maintainable it is and the easier it is to find bugs.

The command line application cloc (Count Lines of Code)\footnote{\url{https://github.com/AlDanial/cloc}} will be used to count the number of source code lines in each project.
An external application will be used to count the number of lines due to the possibility of human error when counting lines in potentially dozens of files.
The version of cloc that will be used is \verb|1.82|.

\Cref{lst:cloc} shows the output of cloc when run against the directory containing the example projects for \cref{sec:comparison}.
The arguments passed to cloc are \verb|--vcs=git| which tells it to use git to list the files in a directory, \verb|--hide-rate| which makes the output of cloc deterministic, and the directory name containing the source code files that need to be counted.
The benefit of indicating that git is the version control system that is being used is that it prevents cloc from counted the lines of non-versioned files.
This prevents build artefacts from being included in the line count.

\begin{lstlisting}[numbers=none,float=h,label=lst:cloc,caption=Output of cloc when run on the example projects]
$ cloc --vcs=git --hide-rate dissertation/examples
      34 text files.
      34 unique files.
      14 files ignored.

github.com/AlDanial/cloc v 1.82
-------------------------------------------------------------------------------
Language                     files          blank        comment           code
-------------------------------------------------------------------------------
Java                             2             26              0             96
Maven                            2             10              0             92
Elixir                           4             19              0             92
Erlang                           4             31              0             81
Python                           3             17              0             79
Markdown                         5             19              0             51
-------------------------------------------------------------------------------
SUM:                            20            122              0            491
-------------------------------------------------------------------------------
\end{lstlisting}

The output of cloc is a table containing columns for each language that is used in the directory, along with the number files of that language, how many blank lines there are, how many comments there are and the actual lines of source code.
For the purpose of this benchmark only the ``code'' column is relevant as this is the number of lines that contain program code.

The directories that will be measured for the two projects will be \verb|src/|\\ \verb|supply_chain_elixir/lib| and \verb|src/supply_chain_jade/src/main/java|.
These directories have been chosen as they only contain the source code for the application instead of also holding configuration files which would unfairly increase the line count.

A way to measure the performance of the system is to measure the CPU usage.
If the CPU usage of the system is low then it means that less pressure is being put on the hardware.
Low CPU usage can come from writing efficient code or from having a runtime that properly spreads the load between all CPU cores.

As the benchmark will be running using Docker, the CPU load may be different if the system were running natively.
This is a limitation of choosing Docker as the way to run these benchmarks, however, as both systems will be running under Docker there should not be any significant differences between them caused by this.

The CPU usage will be measured for each round of the simulation.
Due to measuring the CPU usage of multiple distributed nodes and the fact that nodes can only communicate via message passing, the CPU usage measures will likely be non-deterministic.
A way to ensure accurate results despite this is to run the benchmark multiple times and to calculate the average CPU usage per round for all benchmark runs.

Elixir.
Use Erlang module \verb|cpu_sup|.
Runtime uses one scheduler thread per CPU core, these handle all the processes that are running.

JADE\@.
Unsure what to use at the moment in order to measure the CPU usage.
Each agent runs on a separate OS thread.

One of the ways to measure the speed of the system is to measure the time it takes in order to complete all 220 rounds of the simulation.
This would show the actual speed of the multi-agent system instead of just how much CPU usage it uses.

Both the Elixir and JADE implementation print log messages each time a new simulation round starts.
By looking at when the final round message is printed and subtracting the time when the first round message was printed, we get the total runtime for the simulation.

As a message is printed for each round, it is also possible to work out the time between each round.
This would allow us to calculate the average round time as well as the minimum and maximum round times.
This would be useful to see if each round completes in roughly the same amount of time or whether there are rounds that take significantly longer.

The log messages are printed with millisecond precision time which should be enough precision for this benchmark.
It would be possible to get nanosecond precision for the Elixir system by calling the Erlang standard library time functions and printing that separately.
Unfortunately, nanosecond precision would not be possible for the JADE system.
This feature was only made available in JDK 9 onwards, while this project is using JDK 8.

The final way that the effectiveness of the two implementations will be measured is via their memory consumption.
Lower memory consumption is important as it means that less expensive hardware would be required in order to run a system.
By measuring the increase in memory usage over the rounds in the simulation we can see if there are any issues with how much information the agents remember between rounds.
If the memory usage increase between each round is substantial, the agents could be tuned to forget stale information faster.

Elixir.
Use Erlang module \verb|memsup|.
Maybe look into how Erlang GC works?

JADE\@.
Unsure how to measure memory usage in Java.
Look at Java GC in JDK 8?


\subsection{Experiments}

In order to be able to compare the two systems, a set of experiments need to be defined so that it is possible to see how the systems perform in different scenarios.
It is important to use multiple scenarios as this would find potential bottlenecks that might not be visible in only a simple experiment.
This is also relevant as in the real world a multi-agent system, especially one for supply chains would need to be able to handle varied and dynamic environments.
For simplicity, all the experiments run will be static.
This means that no new agents will intentionally join or leave the simulation while it is running.

The variables that will be adjusted in each experiment are, the number of producer agents that are available, the base goods that are available to be produced and the recipes for what PCs can be manufactured.
This results in experiments that increase the distributed complexity of the simulation.
This direction of experimentation was chosen over making the behaviour of each agent more complex, as it scales better and could provide more insightful results for real-world systems.
This is because if only the complexity of agent behaviours increase they may end up being over-optimised for a single scenario and provide less general results.
Ideally, each scenario should be run with a set of simple and complex behaviours but this is not possible in the time provided.

The first experiment is a simple experiment where there is only one Producer agent which produces a single component type.
This good type can be directly converted into a manufactured PC at a one to one rate.
This is a very simple experiment as there are no decisions that need to be made about what goods should be purchased and what types of PCs need to be manufactured.
The performance in this experiment would be considered a best-case scenario.

The second experiment is similar to the first experiment but that there are now two Producer agents that produce this single component type.
This means that while the act of manufacturing PCs is still simple given that there is only one recipe to do so, the Manufacturer agent now needs to make decisions about where it should buy components from.
This is now a slightly more realistic example as in real supply chains there will be multiple producers of the same kind of component.

The third experiment changes the recipe for the PC production to use two different component types.
There are two Producer agents in this scenario, one for each component type.
Now instead of having to decide where to buy a single component, the Manufacturer agent needs to handle buying different components from different places.
This is again coming closer to a realistic scenario as the manufacturing of computers requires multiple different components.

The fourth experiment combines aspects of the second and third experiment by using a two-component PC recipe where each component has two Producer agents that produce it.
This now requires the Manufacturer agent to both handle multiple component types as well as being able to procure them from different agents.

The final experiment is based on the scenario from TAC SCM but has been simplified to avoid adding extra development effort.
The TAC SCM scenario has multiple producers for CPUs, motherboards, memory and hard drives, and each producer can create two types of components.
Additionally, CPU types need to be paired up with specific motherboard types in order to be compatible.
This level of complexity means that TAC SCM defines 16 valid recipes for producing various types of computers.
Recreating this scenario is a little out of scope for this project as it would require implementing the ability for Producer agents to produce different types of components, which is a feature that would only be required for this benchmark.
Instead, the scenario will be that there are two recipes for producing PCs, one recipe for a ``slow'' computer and another recipe for a ``fast'' computer.
These recipes would require 4 different components where two of them are unique to that recipe.
This experiment will use 8 Producer agents, two for each of the common components and then one Producer agent per unique component.



\section{Results}

This section contains the results and analysis of the 5 experiments that were described in the previous section.
The aim of this section is to answer the research questions that were proposed in the introduction of this dissertation.
The results from running the 5 experiments will be used to determine whether the multi-agent system implemented in Elixir performs better than an identical system written used JADE\@.
The source code for both systems will then be compared to see whether the Elixir system is short and potentially easier to maintain that the JADE version.

The hardware that these benchmarks were run on was an AMD Ryzen 7 3700X processor, running at a base clock speed of 3.6 Ghz and 16 logical processors.
The hardware had 32 GB of DDR4 memory, running at 3533 Mhz.
As the experiments were run under Docker, the results may differ if the experiments were to be run directly on equivalent hardware.
Docker was configured to use all the available CPU cores and to use a maximum of 25 GB of memory.

The host machine for these experiments was running Windows 10 version 21H1 build 19043.1348.
The version of Docker used to run these experiments was Docker Desktop 4.1.1.
The Elixir system was built using the Elixir version 1.12.3 Docker image.
The JADE system was built using the Maven version 3.8.3 OpenJDK 8 image.
Both of these images use Ubuntu as the base operating system.

Each experiment was run 10 times to ensure results were reproducible.
Before each experiment was run, a warmup run was conducted.
This was to ensure that the Docker image was built and that any containers that were required had been instantiated.

\subsection{CPU Usage}

The first benchmark taken was the CPU utilisation of the Manufacturer agent for each experiment.
A higher CPU utilisation means that the CPU is spending less time being idle and shows the proportion of execution time the program takes.
The measurements shown here are the average CPU utilisation over 10 runs of each experiment.

\Cref{fig:mean_cpu} shows the average CPU utilisation for the Elixir and JADE systems in each experiment.
The error bars display the standard deviation from the mean value.
These results show two interesting characteristics.
The first is that Elixir uses close to 100\% CPU utilisation on average for most experiments, while JADE uses less than 20\% for all experiments.
The second is that as the complexity of the experiments increases, both systems use less CPU utilisation.

\begin{figure}[ht]
    \centering
    \includegraphics[width=\textwidth]{mean_cpu.png}
    \caption{The mean CPU utilisation for each experiment}\label{fig:mean_cpu}
\end{figure}

As all the experiments are running under Docker it can be difficult to exactly work out why the systems are behaving in a certain way.
However, educated conjectures can be made as to the cause of these behaviours.

The first interesting characteristic could potentially be caused by the difference in threading models between Elixir and JADE\@.
Elixir by default uses as many scheduler threads as available CPU cores, for these experiments that would be 16 scheduler threads.
Any process that exists in the Elixir system can be scheduled on any of the available scheduler threads.
By contrast, JADE allocates a single dedicated thread to each agent and that agent will only ever execute on that thread.
While the JVM (Java Virtual Machine) allocates some threads for itself, the majority of work being done is happening on three threads in the JADE system.
If we assume that three threads are continuously running on a 16 machine, we can expect the CPU utilisation to be \(\frac{3}{16} \times 100 = 18.75\% \).
This is close to the actual average CPU utilisation of the JADE system.

\begin{figure}[ht]
    \centering
    \includegraphics[width=\textwidth]{cpu_per_round.png}
    \caption{The CPU utilisation per round for run 1 of experiment 5}\label{fig:cpu_per_round}
\end{figure}

\Cref{fig:cpu_per_round} shows the CPU utilisation per round in the first run of the fifth experiment.
We can see that the Elixir system spends almost the entire time at a high CPU utilisation of over 75\% with only a few spikes that are lower.
The JADE system starts with a higher than average CPU utilisation, peaking at 31.25\% utilisation.
If we assume that during the start of the benchmark there will be more load on the general JVM threads, this would explain why the system never comes close to this value of utilisation later on.

The second characteristic where the CPU utilisation decreases with complexity seems likely to be caused by more cross-machine messages being sent.
While this is difficult to verify without looking at the low-level implementation details of how messages are sent in JADE and Elixir, it seems likely that at some point when sending a message to another machine they would need to ask the OS kernel to send a network message.
As the more complex experiments increase the number of agent machines in the network, more messages between machines will be sent.
Since the systems are running in sparse Ubuntu Docker images, this context switching to kernel space to make network requests can have a measurable impact on the CPU utilisation of the system itself.

Looking at only the metric of CPU utilisation in isolation, it would seem as though JADE is better than Elixir as JADE uses few system resources.
However, it can also be interpreted that a low CPU utilisation indicates that the runtime is not using all of the system resources as effectively as it could be.
This result will need to be considered with the other runtime metrics to form a full opinion.


\subsection{Memory}\label{sec:memory}

The second benchmark taken was the average memory usage of the Manufacturer agent for each experiment.
Higher memory usage means that the multi-agent system would need more expensive hardware to run.

\Cref{fig:mean_memory} shows the average memory for the Elixir and JADE systems in each of the experiments.
The error bars indicate the standard deviation of the measurements.

\begin{figure}[ht]
    \centering
    \includegraphics[width=\textwidth]{mean_memory.png}
    \caption{The mean memory usage for each experiment}\label{fig:mean_memory}
\end{figure}

From these results, we can see that the Elixir system uses approximately the same amount of memory regardless of the complexity of the experiment.
The average memory usage for any of the experiments was always between 64 and 65 megabytes with a consistent standard deviation of around 4.15.

However, the JADE system uses more memory as the experiments become more complex, starting with a mean memory usage of 173 megabytes in the first experiment and ending up with a mean memory usage of 254 megabytes in the final experiment.
It can also be seen that the standard deviation increases as the experiments become more complex.
This shows that JADE does not consistently use the same amount of memory but rather frequently has dips or increases in memory usage.
During the first experiment, there was a standard deviation of 74 while in the final experiment there was a standard deviation of 160.

\Cref{fig:memory_per_round} shows the memory usage each round for the first run of the final experiment.
We can see that the Elixir system has only very slight deviations in terms of memory usage.
However, in this chart, we can see where the large standard deviation comes from in the JADE system.

We can see that the memory in the JADE system steadily rises before rapidly being freed.
This is a result of the Java garbage collector running.
We can note that after every two garbage collections the limit of memory that is in use before another garbage collection is called increases.
We can also see that as the experiment runs, garbage collections are called at a more frequent interval.

\begin{figure}[ht]
    \centering
    \includegraphics[width=\textwidth]{memory_per_round.png}
    \caption{The memory usage per round for run 1 of experiment 5}\label{fig:memory_per_round}
\end{figure}

These differences in memory usage can likely be explained by the differing garbage collection strategies in Elixir and JADE, as well as some of the architectural design decisions.

While both Java and Elixir use a generational garbage collector, the Java garbage collector runs globally while the Elixir garbage collector runs on a per-process level.
This means that in Java all the garbage memory will be cleared at once whereas in Elixir only the processes that produce a lot of memory garbage will be required to run their garbage collector.
This along with the use of many more scheduler threads means that the Elixir system can have certain processes currently blocked by garbage collection, however other processes can still run and act on incoming messages.

The Elixir system makes great use of ETS to store information that needs to be accessed by any of the layers in an agent.
This means that there will only ever be one copy of an element in the knowledge base and the element would only need to be cleared if it were explicitly deleted.
While a globally accessible knowledge base would be possible in JADE, this style of coding is discouraged in Java and would require the use of mutex locks to ensure thread safety.
The approach taken in the JADE system was to pass around serialisable objects that store the state of the agent.
However, this means that these objects must be cloned before they are sent to another layer and that when they are filtered during various operations in the Behaviour layer, they must be collected back into new collections.
All of these operations would produce memory garbage.

From these results, we can see that Elixir practically always uses less memory than a system using JADE\@.
While it could be possible to reduce some of the memory allocations in the JADE system, this could require dealing with mutex lock contention and other thread-safety issues which were not required in Elixir.


\subsection{Time}\label{sec:time}

The third benchmark was the mean total time for all of the 220 rounds to finish for each of the experiments.
This measurement shows how quickly the system can make decisions and execute other steps such as sending messages.

\Cref{fig:mean_time_diff} shows the average time that was required for all the experiments to finish.
The error bars show the standard deviation from the mean value.
It is immediately apparent that the JADE system is significantly slower than the Elixir system.
All of the experiments run in under 1 second for the Elixir system with most running under half a second.
The simplest experiment took on average 3 seconds for the JADE system and the most complex experiment took over 12 seconds on average to complete.

\begin{figure}[ht]
    \centering
    \includegraphics[width=\textwidth]{mean_time_diff.png}
    \caption{The mean time taken for each experiment}\label{fig:mean_time_diff}
\end{figure}

Not plotted is the average time between each round.
As each experiment uses the same number of rounds, the average time per round can be found by dividing the total time by 220.

\Cref{fig:time_per_round} shows the time taken between each round for a specific run of experiment 5.
In this chart, we can see from the trendline that the Elixir system has a fairly consistent time per round.
Interestingly, we can see that the JADE system starts slow but speeds up towards the middle of the experiment, before slowing down a bit again near the end.
This is likely caused by Java's JIT compiler which is still optimising the program for the first rounds of the experiment.

\begin{figure}[ht]
    \centering
    \includegraphics[width=\textwidth]{time_per_round.png}
    \caption{The time taken between each round for the first run of experiment 5}\label{fig:time_per_round}
\end{figure}

From these results, we can easily see that Elixir is faster than a system implemented in JADE\@.
This is an unexpected result as it was expected that the maturity of the Java platform would give JADE an advantage in terms of execution time.
Further investigation would be required to work out what is causing the JADE system to perform so slowly.
Whether it is caused by a potential bad design in the multi-agent system or whether the JADE framework code is unoptimised.


\subsection{Lines of Code}

The final metric that was measured was the total numbers of lines of code required in order to implement the multi-agent system.
This can be seen as a measure of how maintainable the code base of such a system would be.
As the fewer lines of code that are required, means that there are less lines of code that need to be checked for bugs.

This is an interesting comparison as Elixir and Java use different programming paradigms.
Java is an object-oriented programming language while Elixir is a functional programming language.
It is expected that Elixir would require less lines of code, however, potentially more functionality may need to be implemented as Elixir lacks a multi-agent framework like JADE\@.

\Cref{tab:elixir_loc} shows the total lines of code that was used to implement the Elixir system, as reported by cloc.
We can see that a total of 1107 lines of source code were used to implement the entire Elixir system.

\begin{table}[h]
    \centering
    \begin{tabular}{lrrrr}
        \toprule
        Language & files & blank & comment & code\\
        \midrule
        Elixir & 19 & 279 & 73 & 1107\\
        \midrule
        SUM\@: & 19 & 279 & 73 & 1107\\
        \bottomrule
    \end{tabular}
    \caption{Lines of code for Elixir system, reported by cloc}\label{tab:elixir_loc}
\end{table}

Table X shows the total lines of code for JADE\@.

Add Table X here.


\subsection{Summary}

From the results gathered by running all five experiments, we can see that Elixir produces a system that is significantly more memory efficient and runs significantly faster.
The downside is that it utilises practically all of the CPU resources available.

The high CPU utilisation may be a product of the fact that in these experiments after one round finishes, the next round is immediately started.
This provides the system with no downtime, resulting in all of the 16 scheduler threads being constantly active.
If these experiments were modelled more closely to the round structure of TAC SCM where each round always takes 15 seconds, it is likely that the Elixir CPU utilisation may be closer to the JADE values.
A multi-agent system where each round always starts immediately after the last round is not very representative of a real-world system which would be closer to a real-time system with periods of downtime.

The consistently low memory usage of the Elixir system would be a benefit for using the language in a real-system.
It would mean that an agent in a supply chain could be hosted on cheaper hardware and would not require a larger memory capacity just to deal with spikes in memory usage.

The low execution time for the Elixir system would also be a benefit in a real-world system, as it indicates that the agent would be able to make decisions faster.
Being able to make a decision faster means that a business plan can be put in place or adjusted quickly while the information the agent received is still current and accurate.

The Elixir system was also found to be shorter in terms of lines of code than the JADE system.
This shows that using a dedicated framework for agent-based programming may potentially not help in implementing a concise system.
As systems written in Elixir are shorter it would allow them to be maintained with greater ease.
This would help in a real-world system where the behaviour of an agent may need to be modified after it has been deployed.


\section{Conclusion}

Using the results from \cref{sec:results}, the two research questions introduced in \cref{sec:research_questions} will now be answered.

When looking at the extent to which Elixir is easier to use when writing a multi-agent system compared to JADE, the empiric metric chosen was the number of lines of code that were required.
The results in \cref{tab:loc} show that Elixir required 495 fewer lines of code compared to JADE\@.
This results in around a third reduction in lines of code compared to JADE\@.
This shows that the provided OTP behaviours in Elixir can be easily used for agent-based design in a multi-agent system.
Furthermore, it shows that a lack of a standardised message format such as FIPA ACL does not hinder the design of a multi-agent system in Elixir.
Therefore, the answer to this research question would be that it is easier to write a multi-agent system in Elixir compared to JADE\@.

Using the results in \cref{sec:cpu_usage,sec:memory,sec:time} we can answer to what extent a system written in Elixir performs better than a system written using JADE\@.
We can see that Elixir performs better in terms of memory usage and the total time taken for an experiment, but performs worse in CPU utilisation when compared to JADE\@.
The CPU utilisation result is surprising but seems to be due to the difference in runtime compared with JADE, as well as the fact that the experiments provide no downtime for the agents.
Arguably, the most important metric is the time taken for the experiments.
This is because, in a real-world system, agents that perform their tasks as quickly as possible would be performed.
For this metric Elixir performed between 1 and 2 orders of magnitude better than the equivalent JADE system.
The answer to this research question would be that for the most part, a system written in Elixir performs better than a system written in JADE\@.

The answers to both of the research questions asked in this dissertation indicate that Elixir could be a viable alternative instead of using JADE for multi-agent systems.
This is due to it both requiring fewer lines of code to implement a multi-agent system, as well as performing significantly better than a system using JADE\@.

\subsection{Reflection}

While this project was able to answer the research questions that were set in the beginning, there were still some major shortcomings.
The largest shortcoming was the lack of complexity in the decision making of the individual agents in the system.

While the agents can buy and sell their goods, a lot of planned strategic decision making had to be cut due to time constraints.
The agents currently only seek to sell their goods for an immediate amount of profit, with no behaviour to plan for the future.
This can be seen with the amount of money that the Manufacturer agent makes.
Either the Manufacturer agent makes a nice profit or spends the entire experiment haemorrhaging funds.
Similarly, there is a lack of handling for exceptional circumstances such as if an order were to arrive late or not at all.
Implementing more complex behaviour might have narrowed the performance gap between Elixir and JADE as Elixir is not a language where the performance of `number-crunching' tasks is prioritised, whereas in Java these types of workloads are more common.

This leads to the second shortcoming which is a lack of detailed performance analysis.
While the results measured show that JADE is significantly slower than Elixir, they don't show why JADE might be slower than Elixir.
This could be investigated by the use of detailed profiling.
This would allow us to see where the JADE system was spending most of its time, and whether this was something that could be fixed in the agent.

If performance issues were to be found in the JADE framework itself, it might be difficult for a fix to be applied.
As mentioned in the literature review, development on JADE has affectively halted since 2017.
During the writing of this dissertation, the original JADE website and project ended up being taken down.
Currently a fork of JADE is being maintained by C\'edric Herpson\footnote{\url{http://www-desir.lip6.fr/~herpsonc/en/}} at Sorbonne University, France.\footnote{\url{https://jade-project.gitlab.io/}}
It has yet to be seen how this will impact the use of JADE for multi-agent systems, but it highlights the importance of evaluating alternate technologies for research.

\subsection{Future Work}

Building off this dissertation, there are two avenues in which future work can be done.

The first would be to cooperate with other researchers in the field of multi-agent systems to determine what functionality they would require from Elixir.
This would help gauge the community interest in using Elixir for multi-agent systems.
It would also hopefully provide a list of features that Elixir would need to implement to be used in research.
As Elixir is a general-purpose programming language, these features would not likely be implemented at the language level.
However, Elixir supports the easy use of software libraries and has a centralised package repository.
This would allow any required functionality such as a FIPA compliant message protocol to be implemented as a library and then distributed as open-source software.

The other avenue would be to investigate other features of Elixir to see how they could be used when implementing multi-agent systems.
Currently, this dissertation mostly uses the features of OTP, built-in support for distributed systems and some of the support for concurrent processes.
However, a system could be designed that uses these features to their full extent as well as investigating how the language support for fault tolerance and hot-code reloading could be applied to multi-agent systems.
This system could also look at using the web-framework Phoenix\footnote{\url{https://www.phoenixframework.org/}} to provide a method of human-agent communication.


\setlength{\bibitemsep}{0.5\baselineskip}
\interlinepenalty=10000 % Disallows page breaks from occurring in a line

\bibliographystyle{apacite}
\bibliography{bibliography}

\pagebreak
\begin{appendices}
\section{Project Overview}

\includepdf[pages=-, pagecommand={\thispagestyle{empty}}]{../management/ipo.pdf}

\section{Second Formal Review Output}
Insert a copy of the project review form you were given at the end of the review by the second marker

\section{Diary Sheets}

\includepdf[pages=-, pagecommand={\thispagestyle{plain}}]{../management/project_diary/week1.pdf}
\includepdf[pages=-, pagecommand={\thispagestyle{plain}}]{../management/project_diary/week2.pdf}
\includepdf[pages=-, pagecommand={\thispagestyle{plain}}]{../management/project_diary/week3.pdf}
\includepdf[pages=-, pagecommand={\thispagestyle{plain}}]{../management/project_diary/week4.pdf}

\end{appendices}


\end{document}
