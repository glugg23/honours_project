\documentclass[12pt,a4paper]{article}
\usepackage{titlesec}
\usepackage{fancyhdr}
\newcommand{\sectionbreak}{\clearpage}
% !TEX encoding = UTF-8 Unicode
\usepackage[utf8]{inputenc} % set input encoding (not needed with XeLaTeX)
\usepackage{graphicx} % support the \includegraphics command and options
\usepackage{pdfpages} % use for the \includepdf command
\usepackage{pax} % use for keeping any hyperlinks when using \includepdf

\usepackage[parfill]{parskip} % begin paragraphs with an empty line rather than an indent
\usepackage{booktabs} % for much better looking tables
\usepackage{array} % for better arrays (eg matrices) in maths
\usepackage{paralist} % very flexible & customisable lists (eg. enumerate/itemize, etc.)
\usepackage{enumitem} % allows you to add styles to lists
\usepackage{verbatim} % adds environment for commenting out blocks of text & for better verbatim
\usepackage{subfig} % make it possible to include more than one captioned figure/table in a single float
\usepackage[toc,page]{appendix}
\usepackage{listings} % use for code listings
\usepackage[hidelinks]{hyperref} % for clickable links, hide the colour boxes surrounding the links
\usepackage{apacite} % needs to be loaded here or hyperref breaks things
\usepackage{cleveref} % use to reference multiple things at once, load here or it breaks

\pagestyle{fancyplain}
\fancyhf{}
\renewcommand{\headrulewidth}{0.5pt}
\renewcommand{\footrulewidth}{0.5pt}
\setlength{\headheight}{15pt}
\fancyhead[L]{Johann Leonhardt --- 40315336}
\fancyhead[R]{SOC10101 Honours Project}
\fancyfoot[L]{}
\fancyfoot[C]{\thepage}

\lstdefinestyle{code}{
    numbers=left,
    breaklines=true,
    showstringspaces=false,
    captionpos=b,
}
\lstset{basicstyle=\tiny, style=code}

\graphicspath{{./img/}}

\begin{document}

\input{title.tex}
\pagebreak
\input{declaration.tex}
\pagebreak
\input{data_protection.tex}
\pagebreak

\begin{abstract}

A multi-agent system is a computer system where multiple intelligent agents interact with each other using messages to solve a common goal.
Most multi-agent systems have been developed using a Java framework called JADE\@.
This dissertation evaluates whether the programming language Elixir, which has a lot of features that can easily be used for agent-based design, performs better than JADE at implementing a multi-agent system.
The theme of using a supply chain system was chosen due to the popularity of the TAC SCM competition.
This dissertation measures the lines of source code required to implement a system in Elixir against the amount required for JADE\@.
It also measures the CPU utilisation, memory usage and the time taken to complete an experiment for both systems.
After completing five experiments this dissertation finds that Elixir performs better on all metrics apart from CPU utilisation.
This dissertation concludes that Elixir is a viable choice for implementing multi-agent systems.

\end{abstract}
\pagebreak

\tableofcontents
\newpage

\listoffigures
\newpage

\renewcommand\lstlistlistingname{List of Listings}
\lstlistoflistings{}
\newpage

\listoftables
\newpage

\section*{Acknowledgements}

Many thanks to my supervisor, Dr Simon T. Powers, for helping me take my vague initial idea and actualise this dissertation.

Dedicated to the work of Joe Armstrong and Fabio Bellifemine.

\newpage

\section{Introduction}

This project aims to see how effective it is to write a multi-agent system in \citeA{elixir2021elixir}.

A multi-agent systems consists of several intelligent agents that interact with each other in order to achieve either common or conflicting goals.
In complex systems, agents can be used instead of humans for decision making.
One of these areas that this project will be looking at is the management of supply chains.
Supply chains have multiple individual parts to them, each of which is trying to maximise their own profits.
If this can be done with a multi-agent system then decisions can be made more rapidly which could improve the supply chain as a whole.

Elixir is a functional programming language that uses the Erlang VM\@.
This gives Elixir a lot of advantages for writing fault tolerant distributed systems.
Such as lightweight threads which do not share state and can only communicate using messages.
Elixir could be considered a good technology to use for implementing multi-agent systems as the concepts behind agent-based design can be easily mapped to existing functionality in the language itself.

Most multi-agent systems have been implemented using a framework called JADE\@.~\cite{bellifemine1999jade}
This is a Java framework that allows you to create FIPA compliant agent systems.
This project tries to see whether Elixir as a language is better suited for writing agents or whether the specification compliance of JADE is enough to ensure it is a good choice.

The aims of this project is to implement a multi-agent system in Elixir and then to reimplement the same system using JADE\@.
After this is done, both of these systems can be compared to find which one is superior.

This project could be important for other computer scientists studying multi-agent systems as Elixir has not yet been used in this area.
If the Elixir implementation is more performant and easier to write than a version written in JADE, it could show that more modern technologies should be considered over using JADE\@.

\subsection{Deliverables}

There will be three deliverables for this project.
The first deliverable will be the complete multi-agent benchmarking system written in Elixir.
The second deliverable will be the comparison benchmarking system written in JADE\@.
The final deliverable will be the results of the two benchmark systems.

\subsection{Research Questions}

This project seeks to answer the following research questions:

\begin{enumerate}
    \item Compared to JADE, to what extent is it easier to write a multi-agent system in Elixir.
    \item Does a system written in Elixir perform better than a system written in JADE\@.
\end{enumerate}

The first question aims to see whether a system implemented in Elixir is more succinctly written than an equivalent system implemented using JADE\@.
The second question tries to see whether a system written in Elixir performs better in terms of CPU and memory usage than a system written with JADE\@.

\subsection{Project Overview}

This project will have several milestones that need to be completed.

\begin{description}[style=nextline]
    \item [Literature Review] Background research on existing multi-agent systems primarily looking at supply chains and the technology that they use will need to be done to provide a foundation for the implementation of this project.
    \item [System Design] The multi-agent system will need to be designed so that it can be implemented in both Elixir and JADE, and distributed across multiple machines\@.
    \item [Implementation] After the system has been designed it will need to be implemented in both Elixir and JADE\@.
    \item [Analysis] Once the systems have been fully implemented they can be benchmarked and compared to see if Elixir produces a better multi-agent system.
\end{description}


\section{Literature Review}

Introduction here.

\subsection{Modelling Supply Chains}

\subsection{Agent-based Design}

\subsection{Technologies for Agent-based Modelling}

\subsubsection{Erlang}

The Erlang\footnote{\url{https://www.erlang.org/}} programming language was introduced by \citeA{armstrong1990erlang} as an experimental new programming language to be used for programming telephony systems.
Erlang is designed around using concurrent processes with no shared memory, which communicate using asynchronous message passing.
All of this is built into the language definition.~\cite{armstrong2007history}
The fact that this behaviour is implemented on a language basis and not as a third-party library is beneficial as it allows us to implement the basic Actor principle of message passing entirely using Erlang processes.

\citeA{varela2004modelling} describes a method for implementing agents using multiple Erlang processes to represent individual components of an agent.
These processes communicate with one another to execute overall actions for the agent, and one or more processes are dedicated to communicating with other agents in the system.
They also describe the notion of a supervisor tree, which is a tree-like hierarchy of agents where each agent monitors the state of any agents below it.
This allows fault tolerance as long as the supervisor stores the initial state of all the agents it monitors.

\citeA{di2005using} evaluated Erlang on three metrics: Agent Model Compliance, Support for Rationality and Support for Distribution.
They found that while Erlang fully met the criteria for distribution support, it did not fully support agent behaviour composition and extension, which would be required for agent model compliance.
They also found that Erlang lacked logical inference support to provide rationality to agents.
They propose the eXAT (\textbf{e}rlang e\textbf{X}perimental \textbf{A}gent \textbf{T}ool) platform as a suitable library for fixing these issues.~\cite{di2003exat}
However, \citeA{peregud2012implementing} found that the eXAT platform was cumbersome to write models with, was practically discontinued since 2005 and has no community support and no production systems using it.

\citeA{santana2017interscsimulator} have used Erlang to model a large scale traffic simulation using millions of agents.
They note that the parallelism provided by Erlang processes and the ability to run Erlang code distributed on multiple machines simultaneously are the main factors for the performance they achieved.
However, they mention that the drawbacks of Erlang are a lack of process synchronisation mechanisms and a general lack of IDE tooling.

While Erlang has been used for implementing other types of scientific modelling, it has yet to be used for modelling supply chains.

\subsubsection{Elixir}

The Elixir\footnote{\url{https://elixir-lang.org/}} programming language can be considered to be a more modern version of Erlang.
Elixir compiles to the same byte code that the Erlang runtime uses.
This allows Elixir code to call code written in Erlang with no runtime overhead.
Elixir uses a syntax similar to Ruby, which is more modern than the Prolog inspired syntax of Erlang.~\cite{loder2016erlang}

As Elixir is a fairly new programming language it has yet to be used for any type of modelling research.
\citeA{fedrecheski2016elixir} compared Elixir with Java for use in Internet of Things software.
They found that while the Java implementation used 5\% to 15\% less CPU load, the Elixir version was shorter in terms of lines of code, consumed significantly less memory, and handled HTTP responses better under heavy load.

While Elixir has not yet found much use in the scientific community, it has been used for web development and real-time applications.
Discord had been an early adopter of Elixir for its messaging platform and was able to scale its application to handle five million concurrent users using it.~\cite{vishnevskiy2017discord}
\citeA{pinterest2017introducing} have used Elixir to implement a notification system.
They found that the Elixir version had ten times fewer lines of code than the original Java version, and ran faster as well.

\subsubsection{JADE}

JADE (\textbf{J}ava \textbf{A}gent \textbf{DE}velopment Framework)\footnote{\url{https://jade.tilab.com/}} is a Java framework for developing agent-based applications, and was one of the first FIPA specification compliant frameworks.~\cite{bellifemine1999jade}
JADE provides a platform for executing agents via composable behaviours and allowing both local and distributed agents to communicate with each other.
One optimisation that JADE provides is that FIPA ACL messages are transported between JADE agents on the same platform as Java objects, instead of needing to be serialised into a string format.
For agents on different platforms the message is converted as required but this provides a performance boost for local agents.

\citeA{bergenti2020first} note that in the 20 years that the JADE framework has been available, it has been extensively used in academia for software agent research as well as being used to introduce students to agent-oriented programming.
However, \shortciteA{bergenti2020first} also mention that students using JADE struggle with the complexity of the framework, if they do not have any prior knowledge of software agents.
They propose Jadescript as a language to provide a dedicated syntax for constructing and using JADE agents.

JADE has been used by \citeA{podobnik2006crocodileagent} for implementing an agent to compete in TAC SCM\@.
As they were using an IKB agent model, they mention that a benefit of JADE was that they could separate each layer onto a different computer.
While this results in a more complex system due to requiring lots of intercommunication, they still managed to reach the semi-finals of the competition once and the quarter-finals twice.~\cite{collins2009flexible}

\subsubsection{NetLogo}

NetLogo\footnote{\url{https://ccl.northwestern.edu/netlogo/}} is a multi-agent programming language which provides an environment for modelling complex systems over time
It has been designed for use in both education and research, and since it has been written using Java it can be run on any major operating system.
NetLogo uses agents called `turtles' to move over and interact with a grid of `patches'.
Both of these can be programmed to have certain behaviours.
NetLogo uses a primarily graphical interface to make it easy to view and modify the simulation while it is running.~\cite{tisue2004netlogo}

While NetLogo is popular in general for agent-based modelling, only a few SCM models have used it.
\citeA{arvitrida2015competition} used NetLogo to model the effect of competition and collaboration on supply chain performance.
They selected NetLogo for being relatively simple but still providing all the features that they needed for this model.

\subsubsection{SARL}

SARL\footnote{\url{http://www.sarl.io/}} is a general-purpose agent-based programming language that aims to be both platform and architecturally agnostic.
SARL provides native support for agent-oriented first-class abstractions but does not force the programmer to use them in any specific way.
The reason why SARL is designed to be agnostic is the belief that a more general and less research focused or theoretical approach is required for agent-oriented programming to make a more significant impact on mainstream software engineering.
SARL compiles to Java byte code and can be run on the Janus platform, although other platforms can be used as well.~\cite{rodriguez2014sarl}

As SARL is a relatively new agent-oriented programming language, it has yet to be used for SCM and has also not seen much use for general agent-based modelling yet.

\subsubsection{Comparison Between Technologies}

This section compares all the previously mentioned technologies, in order to evaluate them for this project.
A short example program has written for each technology, so that the ease of writing a multi-agent system can be objectively compared.
See \Cref{tab:tech_comparison} for an overview of the comparison.

The example multi-agent model that was written for each technology is a `counter agent' program.
This program has two agents, the first agent discovers the second agent and then counts to three before telling the other agent to terminate.
This is a very simple example of multiple agents interacting with each other, and is designed to show how much support a given technology has for agent-oriented primitives such as message passing and defining repeating actions.
The lines of code for these programs has been counted as the number of lines not blank or containing a comment.

\Cref{lst:erl_app,lst:erl_sup,lst:erl_counter,lst:erl_partner} shows the counter agent example written in Erlang.
This example uses the Open Telecom Platform (OTP) generic server behaviour to implement message receiving and agent behaviour.
This allows the agents to be managed by an OTP process supervisor, as seen in \cref{lst:erl_sup}.
The benefit of this is that agents that crash can then be automatically restarted by the supervisor.
This example program comes to 81 lines of code.

eXAT has been compared with Java and JADE by \citeA{di2005using}.
They mention that since Java does not support function clause pattern matching and requires the use of objects to provide a symbolic language, eXAT is superior for implementing and agent-based model.
While \shortciteA{peregud2012implementing} is overall critical of eXAT, they mention a benefit of it is that it allows a large amount of eXAT agents to run at once regardless of size.
Compared to JADE which suggests the use of fewer but larger agents in general.
It should be noted that both of these positives are a result of the Erlang language and not something entirely specific to the eXAT platform.

It was not possible to write a counter agent example in eXAT\@.
The platform depends on deprecated dependencies causing it to fail to compile on \verb|Erlang/OTP 23.2.6|.
As a result of this and the fact that the last update to eXAT was done by \shortciteA{peregud2012implementing} in 2012, this project seems to be completely inactive.

\Cref{lst:ex_app,lst:ex_counter,lst:ex_partner} shows the counter agent example in Elixir.
This program is implemented in 71 lines of code.
Comparing this to the Erlang program we can see that Elixir allows us to write code that is almost semantically equivalent to Erlang, just using a more modern syntax.
We can see the Erlang compatibility of Elixir by the call to \verb|:timer.send_interval/2| on line 25 of \cref{lst:ex_counter}.
This is calling a function in the \verb|timer| module of the Erlang standard library to send a message every second.
One of the reasons why the Elixir code is shorter than the Erlang code is due to Elixir's use of macros for metaprogramming.
The Elixir statement \verb|use GenServer| is a macro which provides default implementations for functions such as \verb|handle_call/3| and \verb|handle_cast/2| which need to be implemented in Erlang manually (see lines 43--47 in \cref{lst:erl_counter}).

\citeA{christos2016agent} have compared the experience writing a supply chain model in both NetLogo and JADE\@.
They found that NetLogo not requiring an external IDE for programming and having easily embeddable graphical controls made it easy to use.
However, they conclude that JADE programs provide more flexible and robust programs assuming the user is willing to spend the time to learn the technology.

\Cref{lst:netlogo} shows the counter agent example implemented in NetLogo.
This is the shortest implementation of the counter agent at only 32 lines of code.
This conciseness however comes at a price.
NetLogo does not have a native way to send messages between `turtles' and has no way for a `turtle' to refuse to do an action it has been asked to do.
While this makes example programs short and easy to read, it would likely cause issues as models grow larger and more complex.
NetLogo does have a library\footnote{\url{https://users.uom.gr/~iliass/projects/NetLogo/}} to support sending messages between agents, however, this library only supports \verb|NetLogo 5| whereas the current version is \verb|NetLogo 6.2|.~\cite{sakellariou2008enhancing}

For this project I have chosen to use Elixir for implementing my supply chain model.
I believe that the benefits of using a language that can leverage the features of Erlang while simultaneously having a more modern syntax, outweighs the potential issues of using a language that has not yet been used for research purposes.

\begin{table}[h]
    \centering
    \caption{Technology Comparison}\label{tab:tech_comparison}
    \begin{tabular}{@{}lllll@{}}
        \toprule
        Name & SCM & FIPA & Difficulty & Active Project\\
        \midrule
        Erlang & No & No &  & Yes\\
        eXAT & No & Yes & N/A & No\\
        Elixir & No & No &  & Yes\\
        JADE & Yes & Yes &  & No?\\
        NetLogo & Yes & Plugin &  & Yes\\
        SARL & No & No &  & Yes\\
        \bottomrule
    \end{tabular}
\end{table}


\section{Methodology}

\subsection{Elixir}

Introduction to Elixir system.

Explain GenServer and how it is applies to agents.

Explain GenServer client/server API model.

Required implementing own message format.

Agent type loaded in via system environment.

Config loaded in via Elixir script file.

\subsubsection{Information Layer}

Handles inter-node connections by listening for nodeup events.

Library ``libcluster'' is used to automatically connect to other agents on the same network.

Builds the information filter for the agent.

Stores the node information in ETS, explain what ETS is.

Forwards any messages that pass through the information filter to the Knowledge layer.

\subsubsection{Knowledge Layer}

Loads config values into knowledge base.
Knowledge base is also stored in ETS\@.

Is responsible for updating Knowledge base as messages are received.

Each agent uses a separate Knowledge layer implementation.
Common functionality is included via the use of macros.

\subsubsection{Behaviour Layer}

The proactive and reactive behaviour of the agent.

Is implemented as a finite state machine using the library ``GenStateMachine'' which is a wrapper class for the Erlang module \verb|gen_statem|.

Explain the behaviour of the different agent types.

\subsection{JADE}

Introduction to JADE system.

Using Java 8 as this is the last long term support version where JADE can be compiled, although due to the nature of Java bytecode a later version could be used.

Each layer is a separate agent instance.

At start-up the specific layers are started depending on the system environment.

\subsubsection{Information Layer}

Passes messages to Knowledge layer.

Is implemented in a similar style to a GenServer, this is done by extending the Cyclic Behaviour from JADE and implementing that in the class.

\subsubsection{Knowledge Layer}

Acts as the knowledge base in the system.

There is an abstract Knowledge agent which the others extend.
This agent defines common behaviour like the Elixir macros.

This class loads config values from a properties file.

Write about attempt to form agent federation and why this did not work.

Write about de-serialising agent information from XML file and why JADE custom collections made serialising JADE objects more difficult.

\subsubsection{Behaviour Layer}

Implemented using JADE's finite state machine behaviour.

Explain the behaviour of the different agent types.

\subsection{Docker}

Explain the use of Docker during development.

Explain how Docker could be used for benchmarking.

\subsection{Benchmarking}

Explain the different metrics that will be used for the experiments.

Source lines of code, measured using \verb|cloc|.

CPU usage.

Memory usage.

Time for all the rounds of the simulation?

\subsection{Experiments}

Describe the different experiment setups.

From having 1 generic producer to the full TAC SCM setup.


\section{Results}

Introduce Results section.

Mention hardware that benchmarks were performed on.

Mention software versions that benchmarks were performed on.

Mention how the benchmarks were done.
One warmup round to build images followed by 10 measured rounds.

\subsection{CPU Usage}

Display results for average CPU usage per experiment.


\subsection{Memory}

Display results for average memory usage per experiment.


\subsection{Time}\label{sec:time}

The third benchmark was the mean total time for all of the 220 rounds to finish for each of the experiments.
This measurement shows how quickly the system can make decisions and execute other steps such as sending messages.

\Cref{fig:mean_time_diff} shows the average time that was required for all the experiments to finish.
The error bars show the standard deviation from the mean value.
It is immediately apparent that the JADE system is significantly slower than the Elixir system.
All of the experiments run in under 1 second for the Elixir system with most running under half a second.
The simplest experiment took on average 3 seconds for the JADE system and the most complex experiment took over 12 seconds on average to complete.

\begin{figure}[h]
    \centering
    \includegraphics[width=\textwidth]{mean_time_diff.png}
    \caption{The mean time taken for each experiment}\label{fig:mean_time_diff}
\end{figure}

Not plotted is the average time between each round.
As each experiment uses the same number of rounds, the average time per round can be found by dividing the total time by 220.

\Cref{fig:time_per_round} shows the time taken between each round for a specific run of experiment 5.
In this chart, we can see from the trendline that the Elixir system has a fairly consistent time per round.
Interestingly, we can see that the JADE system starts slow but speeds up towards the middle of the experiment, before slowing down a bit again near the end.
This is likely caused by Java's JIT compiler which is still optimising the program for the first rounds of the experiment.

\begin{figure}[ht]
    \centering
    \includegraphics[width=\textwidth]{time_per_round.png}
    \caption{The time taken between each round for the first run of experiment 5}\label{fig:time_per_round}
\end{figure}

From these results, we can easily see that Elixir is faster than a system implemented in JADE\@.
This is an unexpected result as it was expected that the maturity of the Java platform would give JADE an advantage in terms of execution time.
Further investigation would be required to work out what is causing the JADE system to perform so slowly.
Whether it is caused by a potential bad design in the multi-agent system or whether the JADE framework code is unoptimised.


\subsection{Lines of Code}

Display results for lines of code in Elixir and JADE systems.


\subsection{Conclusion}

Sum up all the results and determine which system is better.


\section{Conclusion}

Using the results from \cref{sec:results}, the two research questions introduced in \cref{sec:research_questions} will now be answered.

When looking at the extent to which Elixir is easier to use when writing a multi-agent system compared to JADE, the empiric metric chosen was the number of lines of code that were required.
The results in \cref{tab:loc} show that Elixir required 495 fewer lines of code compared to JADE\@.
This results in around a third reduction in lines of code compared to JADE\@.
This shows that the provided OTP behaviours in Elixir can be easily used for agent-based design in a multi-agent system.
Furthermore, it shows that a lack of a standardised message format such as FIPA ACL does not hinder the design of a multi-agent system in Elixir.
Therefore, the answer to this research question would be that it is easier to write a multi-agent system in Elixir compared to JADE\@.

Using the results in \cref{sec:cpu_usage,sec:memory,sec:time} we can answer to what extent a system written in Elixir performs better than a system written using JADE\@.
We can see that Elixir performs better in terms of memory usage and the total time taken for an experiment, but performs worse in CPU utilisation when compared to JADE\@.
The CPU utilisation result is surprising but seems to be due to the difference in runtime compared with JADE, as well as the fact that the experiments provide no downtime for the agents.
Arguably, the most important metric is the time taken for the experiments.
This is because, in a real-world system, agents that perform their tasks as quickly as possible would be performed.
For this metric Elixir performed between 1 and 2 orders of magnitude better than the equivalent JADE system.
The answer to this research question would be that for the most part, a system written in Elixir performs better than a system written in JADE\@.

The answers to both of the research questions asked in this dissertation indicate that Elixir could be a viable alternative instead of using JADE for multi-agent systems.
This is due to it both requiring fewer lines of code to implement a multi-agent system, as well as performing significantly better than a system using JADE\@.

\subsection{Reflection}

While this project was able to answer the research questions that were set in the beginning, there were still some major shortcomings.
The largest shortcoming was the lack of complexity in the decision making of the individual agents in the system.

While the agents can buy and sell their goods, a lot of planned strategic decision making had to be cut due to time constraints.
The agents currently only seek to sell their goods for an immediate amount of profit, with no behaviour to plan for the future.
This can be seen with the amount of money that the Manufacturer agent makes.
Either the Manufacturer agent makes a nice profit or spends the entire experiment haemorrhaging funds.
Similarly, there is a lack of handling for exceptional circumstances such as if an order were to arrive late or not at all.
Implementing more complex behaviour might have narrowed the performance gap between Elixir and JADE as Elixir is not a language where the performance of `number-crunching' tasks is prioritised, whereas in Java these types of workloads are more common.

This leads to the second shortcoming which is a lack of detailed performance analysis.
While the results measured show that JADE is significantly slower than Elixir, they don't show why JADE might be slower than Elixir.
This could be investigated by the use of detailed profiling.
This would allow us to see where the JADE system was spending most of its time, and whether this was something that could be fixed in the agent.

If performance issues were to be found in the JADE framework itself, it might be difficult for a fix to be applied.
As mentioned in the literature review, development on JADE has affectively halted since 2017.
During the writing of this dissertation, the original JADE website and project ended up being taken down.
Currently a fork of JADE is being maintained by C\'edric Herpson\footnote{\url{http://www-desir.lip6.fr/~herpsonc/en/}} at Sorbonne University, France.\footnote{\url{https://jade-project.gitlab.io/}}
It has yet to be seen how this will impact the use of JADE for multi-agent systems, but it highlights the importance of evaluating alternate technologies for research.

\subsection{Future Work}

List any future work that could be done.


\setlength{\bibitemsep}{0.5\baselineskip}
\interlinepenalty=10000 % Disallows page breaks from occurring in a line

\bibliographystyle{apacite}
\bibliography{bibliography}

\pagebreak
\begin{appendices}
\section{Project Overview}

\includepdf[pages=-, pagecommand={\thispagestyle{empty}}]{../management/ipo.pdf}

\section{Second Formal Review Output}
Insert a copy of the project review form you were given at the end of the review by the second marker

\section{Diary Sheets}

\includepdf[pages=-, pagecommand={\thispagestyle{plain}}]{../management/project_diary/week1.pdf}
\includepdf[pages=-, pagecommand={\thispagestyle{plain}}]{../management/project_diary/week2.pdf}
\includepdf[pages=-, pagecommand={\thispagestyle{plain}}]{../management/project_diary/week3.pdf}
\includepdf[pages=-, pagecommand={\thispagestyle{plain}}]{../management/project_diary/week4.pdf}

\end{appendices}


\end{document}
