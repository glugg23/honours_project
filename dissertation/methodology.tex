\section{Methodology}

Introduction to the chapter.

Write about what will be covered.

\subsection{System Design}

Write what the system does each round of the simulation.

Write about each type of agent.

Add message sequence diagram for a typical round of the simulation.

\subsection{Agent Design}

Describe each of the layers of the agent and what they do.

Add message sequence diagram for communication between layers.

\subsection{Elixir}

Introduction to Elixir system.

This system makes heavy use of the standard Erlang OTP behaviour\\ \verb|gen_server| (henceforth referred to as GenServer, following standard Elixir naming conventions).
A GenServer is implemented as a Client-Server relationship where a public API is used to interact with user-defined callback functions.
The core behaviour of a GenServer is to wait for a message to be received, then pass that message to the appropriate callback function before looping back to wait for the next message.
This behaviour fits neatly in line with the general behaviour of agents, where the reaction to a message can be implemented as callback functions.

The split Client-Server relationship of a GenServer can be used to define a common interface that all server implementations need to adhere to.
By defining all client-facing methods in a separate module and implementing separate callback functions in different modules, we can guarantee that all particular implementations of an agent can be interacted with in the same manner.
Callback functions can also be implemented without requiring a public interface method.
This allows the GenServer to handle arbitrary messages that have been sent to it.
This functionality has been predominantly used in this system as many messages are being sent between different layers and different network nodes.

While Elixir allows the sending of any data type between different processes, it lacks a defined message type such as the ACLMessage class in JADE\@.
For this system, a message type has been defined by using the struct feature of Elixir.
This means we can define a struct that is similar to a map where only certain keys are allowed.
Certain keys can also be enforced meaning that attempting to initialise a struct with these keys missing would be an error.
The message struct that was defined for this system is based on a subset of fields that are used in FIPA ACL messages.
The message struct contains keys for the performative, message sender, message receiver, whom to reply to, the content of the message and the message ID\@.
Helper functions for replying to, forwarding and sending the message structs have also been implemented.
Not all the messages in this system use this message format.
Where appropriate, such as the callback functions for GenServers use simpler messages such as tuples of primitives.

Instead of creating a separate project for each agent type, a single project is used for all agents.
This is beneficial as all the agent types share a common information layer implementation and reuse common knowledge layer responses.
The type of the agent is set at the start up of the application by reading the system environment variables and checking the value of the \verb|AGENT_TYPE| variable.
This variable indicates which knowledge and behaviour layer implementations should be started.
If this variable is not set or is set to an unrecognised value, then the agent application will exit with an error.

The config for each agent type is loaded in as an Elixir script file, using the Config module in the Elixir standard library.
An Elixir script file is a source code file that is run by being interpreted at runtime instead of being compiled to bytecode.
By importing the Config module, macros can be used to define key-value pairs in a scoped namespace.
This is used to store the initial knowledge for each agent type.
Since this is just an Elixir source code file, any Elixir datatypes can be used.
This means that complex types such as maps or lists of values can be used without having to parse them from a markup language like XML\@.
This is useful as it unifies the language that is used for configuration with the language that is used for code.

\subsubsection{Information Layer}

The Information layer in the Elixir system is used to handle the discovery and connection between all the different nodes in the distributed system.
The library \verb|libcluster|\footnote{\url{https://github.com/bitwalker/libcluster}} is used to handle the automatic discovery of any nodes in the local network.
This library is started in the main application supervisor tree alongside the individual supervisor trees for each layer.
It is configured to use the \verb|Gossip| clustering strategy.
This strategy uses multicast UDP (User Datagram Protocol) to send information packets across the network and forms connections with any nodes that are listening for it.

The Information layer uses the Erlang module \verb|net_kernel| function\\ \verb|monitor_nodes/1| to received messages when a new connection is formed between nodes.
This message takes the form \verb|{:nodeup, node}| where node is the name of the node that was connected to.
Once the Information layer has received this message that a node has connected, it can use the name of the node in order to request information about it.

After the Information layer has requested information about a new node, it adds it to its information filter.
First it looks up if the node is of a type that should be ignored.
If it is it sets a boolean value to be true which indicates that messages from that node should always be ignored.

The information filter is stored as a table in Erlang Term Storage.
Erlang Term Storage (ETS) is a runtime provided, in-memory key-value store.
Each process can create new tables in order to store data.
ETS allows for any type to be stored directly, meaning there is no need to convert between representations.
ETS provides constant time access to data and tables can be configured to be accessible globally.
This makes it ideal for storing information that is required by all of the layers in an agent.

When a message arrives the Information layer checks whether the sender is in the information filter or not.
If the sender is not in the filter, then the message is passed on to the Knowledge layer.
Otherwise, the message is ignored.

\subsubsection{Knowledge Layer}

The Knowledge layer is responsible for loading the initial knowledge into the agent.
It reads a config file which stores all the knowledge that an agent needs to know before a simulation starts.
Then these values are inserted into a table that is stored in ETS\@.
This allows the Behaviour layer to read and update the knowledge base as well.

When the Knowledge layer receives a forwarded message from the Information layer, it gets inserted into an ETS table called the Inbox.
This allows messages to be handled in bulk by the Behaviour layer.

At the end of the simulation the Knowledge layer prints out the total amount of money that an agent has made.

Each agent uses a separate Knowledge layer implementation in order to be able to handle unique message types.
As Elixir lacks a form of class inheritance, macros are used to ensure duplicate behaviour only needs to be written once.

\subsubsection{Behaviour Layer}

The proactive and reactive behaviour of the agent.

Is implemented as a finite state machine using the library ``GenStateMachine'' which is a wrapper class for the Erlang module \verb|gen_statem|.

Explain the behaviour of the different agent types.


\subsection{JADE}

Introduction to JADE system.

Using Java 8 as this is the last long term support version where JADE can be compiled, although due to the nature of Java bytecode a later version could be used.

Each layer is a separate agent instance.

At start-up the specific layers are started depending on the system environment.

\subsubsection{Information Layer}

Passes messages to Knowledge layer.

Is implemented in a similar style to a GenServer, this is done by extending the Cyclic Behaviour from JADE and implementing that in the class.

\subsubsection{Knowledge Layer}

Acts as the knowledge base in the system.

There is an abstract Knowledge agent which the others extend.
This agent defines common behaviour like the Elixir macros.

This class loads config values from a properties file.

Write about attempt to form agent federation and why this did not work.

Write about de-serialising agent information from XML file and why JADE custom collections made serialising JADE objects more difficult.

\subsubsection{Behaviour Layer}

Implemented using JADE's finite state machine behaviour.

Explain the behaviour of the different agent types.

\subsection{Docker}

Explain the use of Docker during development.

Explain how Docker could be used for benchmarking.

\subsection{Benchmarking}

Explain the different metrics that will be used for the experiments.

The first metric that will be measured for this benchmark is the total number of source lines of code.
This will be used to see whether an agent system implemented in Elixir is more concise than a system implemented using JADE\@.
If Elixir is indeed more concise than JADE it would throw into question whether using a framework for implementing multi-agent systems is worth it.
Having a more concise project is beneficial as the less lines of code there are, the more maintainable it is and the easier it is to find bugs.

The command line application cloc (Count Lines of Code)\footnote{\url{https://github.com/AlDanial/cloc}} will be used to count the number of source code lines in each project.
An external application will be used to count the number of lines due to the possibility of human error when counting lines in potentially dozens of files.
The version of cloc that will be used is \verb|1.82|.

\Cref{lst:cloc} shows the output of cloc when run against the directory containing the example projects for \cref{sec:comparison}.
The arguments passed to cloc are \verb|--vcs=git| which tells it to use git to list the files in a directory, \verb|--hide-rate| which makes the output of cloc deterministic, and the directory name containing the source code files that need to be counted.
The benefit of indicating that git is the version control system that is being used is that it prevents cloc from counted the lines of non-versioned files.
This prevents build artefacts from being included in the line count.

\begin{lstlisting}[numbers=none,float=h,label=lst:cloc,caption=Output of cloc when run on the example projects]
$ cloc --vcs=git --hide-rate dissertation/examples
      34 text files.
      34 unique files.
      14 files ignored.

github.com/AlDanial/cloc v 1.82
-------------------------------------------------------------------------------
Language                     files          blank        comment           code
-------------------------------------------------------------------------------
Java                             2             26              0             96
Maven                            2             10              0             92
Elixir                           4             19              0             92
Erlang                           4             31              0             81
Python                           3             17              0             79
Markdown                         5             19              0             51
-------------------------------------------------------------------------------
SUM:                            20            122              0            491
-------------------------------------------------------------------------------
\end{lstlisting}

The output of cloc is a table containing columns for each language that is used in the directory, along with the number files of that language, how many blank lines there are, how many comments there are and the actual lines of source code.
For the purpose of this benchmark only the ``code'' column is relevant as this is the number of lines that contain program code.

The directories that will be measured for the two projects will be \verb|src/|\\ \verb|supply_chain_elixir/lib| and \verb|src/supply_chain_jade/src/main/java|.
These directories have been chosen as they only contain the source code for the application instead of also holding configuration files which would unfairly increase the line count.

A way to measure the performance of the system is to measure the CPU usage.
If the CPU usage of the system is low then it means that less pressure is being put on the hardware.
Low CPU usage can come from writing efficient code or from having a runtime that properly spreads the load between all CPU cores.

As the benchmark will be running using Docker, the CPU load may be different if the system were running natively.
This is a limitation of choosing Docker as the way to run these benchmarks, however, as both systems will be running under Docker there should not be any significant differences between them caused by this.

The CPU usage will be measured for each round of the simulation.
Due to measuring the CPU usage of multiple distributed nodes and the fact that nodes can only communicate via message passing, the CPU usage measures will likely be non-deterministic.
A way to ensure accurate results despite this is to run the benchmark multiple times and to calculate the average CPU usage per round for all benchmark runs.

Elixir.
Use Erlang module \verb|cpu_sup|.
Runtime uses one scheduler thread per CPU core, these handle all the processes that are running.

JADE\@.
Unsure what to use at the moment in order to measure the CPU usage.
Each agent runs on a separate OS thread.

One of the ways to measure the speed of the system is to measure the time it takes in order to complete all 220 rounds of the simulation.
This would show the actual speed of the multi-agent system instead of just how much CPU usage it uses.

Both the Elixir and JADE implementation print log messages each time a new simulation round starts.
By looking at when the final round message is printed and subtracting the time when the first round message was printed, we get the total runtime for the simulation.

As a message is printed for each round, it is also possible to work out the time between each round.
This would allow us to calculate the average round time as well as the minimum and maximum round times.
This would be useful to see if each round completes in roughly the same amount of time or whether there are rounds that take significantly longer.

The log messages are printed with millisecond precision time which should be enough precision for this benchmark.
It would be possible to get nanosecond precision for the Elixir system by calling the Erlang standard library time functions and printing that separately.
Unfortunately, nanosecond precision would not be possible for the JADE system.
This feature was only made available in JDK 9 onwards, while this project is using JDK 8.

The final way that the effectiveness of the two implementations will be measured is via their memory consumption.
Lower memory consumption is important as it means that less expensive hardware would be required in order to run a system.
By measuring the increase in memory usage over the rounds in the simulation we can see if there are any issues with how much information the agents remember between rounds.
If the memory usage increase between each round is substantial, the agents could be tuned to forget stale information faster.

Elixir.
Use Erlang module \verb|memsup|.
Maybe look into how Erlang GC works?

JADE\@.
Unsure how to measure memory usage in Java.
Look at Java GC in JDK 8?


\subsection{Experiments}

In order to be able to compare the two systems, a set of experiments need to be defined so that it is possible to see how the systems perform in different scenarios.
It is important to use multiple scenarios as this would find potential bottlenecks that might not be visible in only a simple experiment.
This is also relevant as in the real world a multi-agent system, especially one for supply chains would need to be able to handle varied and dynamic environments.
For simplicity, all the experiments run will be static.
This means that no new agents will intentionally join or leave the simulation while it is running.

The variables that will be adjusted in each experiment are, the number of producer agents that are available, the base goods that are available to be produced and the recipes for what PCs can be manufactured.
This results in experiments that increase the distributed complexity of the simulation.
This direction of experimentation was chosen over making the behaviour of each agent more complex, as it scales better and could provide more insightful results for real-world systems.
This is because if only the complexity of agent behaviours increase they may end up being over-optimised for a single scenario and provide less general results.
Ideally, each scenario should be run with a set of simple and complex behaviours but this is not possible in the time provided.

The first experiment is a simple experiment where there is only one Producer agent which produces a single component type.
This good type can be directly converted into a manufactured PC at a one to one rate.
This is a very simple experiment as there are no decisions that need to be made about what goods should be purchased and what types of PCs need to be manufactured.
The performance in this experiment would be considered a best-case scenario.

The second experiment is similar to the first experiment but that there are now two Producer agents that produce this single component type.
This means that while the act of manufacturing PCs is still simple given that there is only one recipe to do so, the Manufacturer agent now needs to make decisions about where it should buy components from.
This is now a slightly more realistic example as in real supply chains there will be multiple producers of the same kind of component.

The third experiment changes the recipe for the PC production to use two different component types.
There are two Producer agents in this scenario, one for each component type.
Now instead of having to decide where to buy a single component, the Manufacturer agent needs to handle buying different components from different places.
This is again coming closer to a realistic scenario as the manufacturing of computers requires multiple different components.

The fourth experiment combines aspects of the second and third experiment by using a two-component PC recipe where each component has two Producer agents that produce it.
This now requires the Manufacturer agent to both handle multiple component types as well as being able to procure them from different agents.

The final experiment is based on the scenario from TAC SCM but has been simplified to avoid adding extra development effort.
The TAC SCM scenario has multiple producers for CPUs, motherboards, memory and hard drives, and each producer can create two types of components.
Additionally, CPU types need to be paired up with specific motherboard types in order to be compatible.
This level of complexity means that TAC SCM defines 16 valid recipes for producing various types of computers.
Recreating this scenario is a little out of scope for this project as it would require implementing the ability for Producer agents to produce different types of components, which is a feature that would only be required for this benchmark.
Instead, the scenario will be that there are two recipes for producing PCs, one recipe for a ``slow'' computer and another recipe for a ``fast'' computer.
These recipes would require 4 different components where two of them are unique to that recipe.
This experiment will use 8 Producer agents, two for each of the common components and then one Producer agent per unique component.

