\subsection{Supply Chains}

\subsubsection{Just-in-Time Manufacturing}

This section aims to place the research questions noted in \cref{sec:research_questions} within the broader context of JIT manufacturing.

What is JIT manufacturing?

How can JIT handle changes in demand?

How can JIT handle situational stockpiling?

\subsubsection{Multi-Agent Modelling of Supply Chains}

The Trading Agent Competition for Supply Chain Management (TAC SCM) was a competition with the intent of examining how automated agents perform in the context of dynamic supply chain management.
This was done by having six PC assembly agents competing for raw resources such as CPU and memory components, as well as for the demand requests from customers.~\cite{sadeh2003tac}

The Trading Agent Competition originally started in 2000 as a competition where each agent was a travel agent which was required to organise trips for eight clients and maximise their satisfaction.
In 2002 the success of this competition prompted the organisers to introduce new challenges and this competition would then become TAC SCM in 2003.
The motivations behind TAC SCM were the lack of an existing validation methodology which captured the inherently competitive nature of supply chains.~\cite{arunachalam2005supply}

\textbf{TODO}: More in-depth explanation of how TAC SCM works.~\cite{arunachalam2005supply}

\citeA{collins2010pushing} mentions that TAC SCM has allowed considerable progress to be made for developing strategies and architectures for managing risk in dynamic supply chain environments.
However, these strategies and ideas have not yet been packaged in a way to allow the development of human-oriented decision support tools.

Unfortunately, the TAC SCM competition has not been run since 2011, with PowerTAC becoming the preferred competition for adversarial trading agent research.

While this project is not about designing a competition trading agent and regardless of the fact that TAC SCM is no longer an active competition, it is still important for grounding this project in the existing literature for supply chain modelling.
As a result of this, this project will use a similar theme of PC assembly for the model.

PowerTAC\@.

Other multi-agent models.
