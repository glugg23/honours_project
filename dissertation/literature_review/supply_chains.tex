\subsection{Supply Chains}

\subsubsection{Just-in-Time Manufacturing}

This section aims to place the research questions noted in \cref{sec:research_questions} within the broader context of JIT manufacturing.

What is JIT manufacturing?

How can JIT handle changes in demand?

How can JIT handle situational stockpiling?

\subsubsection{Multi-Agent Modelling of Supply Chains}

The Trading Agent Competition for Supply Chain Management (TAC SCM) was a competition with the intent of examining how automated agents perform in the context of dynamic supply chain management.
This was done by having six PC assembly agents competing for raw resources such as CPU and memory components, as well as for the demand requests from customers.~\cite{sadeh2003tac}

The Trading Agent Competition originally started in 2000 as a competition where each agent was a travel agent which was required to organise trips for eight clients and maximise their satisfaction.
In 2002 the success of this competition prompted the organisers to introduce new challenges and this competition would then become TAC SCM in 2003.
The motivations behind TAC SCM were the lack of an existing validation methodology which captured the inherently competitive nature of supply chains.~\cite{arunachalam2005supply}

TAC SCM was designed so that only trading agents who could coordinate the issues of sourcing, procurement, production and customer bidding would be able to do well.
Each agent needs to bid on RFQs (requests for quotes) from customers to assembly a certain number of PCs within a time frame.
The components for these PCs would be procured from various suppliers of CPUs, motherboards, memory and hard drives via similar RFQs.
Each trading agent has an identical factory which can only operate for a certain number of hours each day, and different PCs require a different amount of time to assemble.
Each agent starts with an empty bank account and needs to initially borrow money from the bank.
The winner of the competition is the agent who has the greatest bank balance at the end of 220 days of production.~\cite{arunachalam2005supply}
Later competitions also added a reputation system between suppliers and trading agents so that subversive market manipulation would be a less viable strategy.~\cite{collins2010pushing}

\citeA{collins2010pushing} mentions that TAC SCM has allowed considerable pro\-gress to be made for developing strategies and architectures for managing risk in dynamic supply chain environments.
However, these strategies and ideas have not yet been packaged in a way to allow the development of human-oriented decision support tools.

Unfortunately, the TAC SCM competition has not been run since 2011, with Power TAC becoming the preferred competition for adversarial trading agent research.

While this project is not about designing a competition trading agent and regardless of the fact that TAC SCM is no longer an active competition, it is still important for grounding this project in the existing literature for supply chain modelling.
As a result of this, this project will use a similar theme of PC assembly for the model.

Power TAC\footnote{\url{https://powertac.org/}} is a trading agent competition that has been running yearly since starting in 2012.
The authors of this competition have built it off of their experience running TAC SCM, and at present it is the primary trading agent competition in the field.~\cite{ketter2013power}

Power TAC was created as a competition due to a lack of other means for experimenting with different market designs and policies regarding retail power production.
Most power girds are managed by a regulated monopoly and lack the necessary monitoring capabilities to be considered a `smart grid'.
Power TAC was created as a low risk way to model potential real world policies for renewable energy markets.~\cite{ketter2013power}

Entries to Power TAC need to implement a broker agent which aggregates energy supply and demand and tries to earn a profit from doing so.
Brokers need to design tariffs and offer these to various types of customers.
Pricing for energy can vary depending on the time of day and the effect of weather.
Each simulation runs for approximately 60 simulated days and at the end of this time the broker agent which has the highest monetary payment minus costs and fees is the winner.~\cite{ketter2020power}

Although Power TAC covers a different domain than this project does, many entries and winners have written papers about their agents and these could be beneficial to see the general patterns of agent design.
