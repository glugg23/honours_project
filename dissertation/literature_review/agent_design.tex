\subsection{Agent-based Design}

\subsubsection{Composition}

Mention IKB, Blackboard System, Holons.

\subsubsection{Communication}

KQML (Knowledge Query and Manipulation Language) is an agent communication language (ACL) developed by the ARPA Knowledge Sharing Effort.
KQML was created as a common standard for sending messages between multiple agents.
It uses an extendable set of performatives to indicate which type of message is being sent, and then the content of the message follows in various fields.
The syntax of the message is based on a balanced parenthesis list similar to the syntax of Common Lisp and can be transmitted as plain text or in a binary form.~\cite{finin1994kqml}

KQML has been criticised by \citeA{cohen1995communicative} for not having a precisely defined set of semantics.
This means when an agent sends a message with a certain performative, there is a no way of knowing if the other agent has implemented the same meaning of that performative.
Another issue they mention is that there is no way in KQML to ask an agent to complete a task in the future.

FIPA\@.

Custom.
