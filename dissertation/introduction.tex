\section{Introduction}

This project aims to see how effective it is to write a multi-agent system in \citeA{elixir2021elixir}.

A multi-agent systems consists of several intelligent agents that interact with each other in order to achieve either common or conflicting goals.
In complex systems, agents can be used instead of humans for decision making.
One of these areas that this project will be looking at is the management of supply chains.
Supply chains have multiple individual parts to them, each of which is trying to maximise their own profits.
If this can be done with a multi-agent system then decisions can be made more rapidly which could improve the supply chain as a whole.

Elixir is a functional programming language that uses the Erlang VM\@.
This gives Elixir a lot of advantages for writing fault tolerant distributed systems.
Such as lightweight threads which do not share state and can only communicate using messages.
Elixir could be considered a good technology to use for implementing multi-agent systems as the concepts behind agent-based design can be easily mapped to existing functionality in the language itself.

Most multi-agent systems have been implemented using a framework called JADE\@.~\cite{bellifemine1999jade}
This is a Java framework that allows you to create FIPA compliant agent systems.
This project tries to see whether Elixir as a language is better suited for writing agents or whether the specification compliance of JADE is enough to ensure it is a good choice.

The aims of this project is to implement a multi-agent system in Elixir and then to reimplement the same system using JADE\@.
After this is done, both of these systems can be compared to find which one is superior.

This project could be important for other computer scientists studying multi-agent systems as Elixir has not yet been used in this area.
If the Elixir implementation is more performant and easier to write than a version written in JADE, it could show that more modern technologies should be considered over using JADE\@.

\subsection{Deliverables}

There will be three deliverables for this project.
The first deliverable will be the complete multi-agent benchmarking system written in Elixir.
The second deliverable will be the comparison benchmarking system written in JADE\@.
The final deliverable will be the results of the two benchmark systems.

\subsection{Research Questions}

This project seeks to answer the following research questions:

\begin{enumerate}
    \item Compared to JADE, to what extent is it easier to write a multi-agent system in Elixir.
    \item Does a system written in Elixir perform better than a system written in JADE\@.
\end{enumerate}

The first question aims to see whether a system implemented in Elixir is more succinctly written than an equivalent system implemented using JADE\@.
The second question tries to see whether a system written in Elixir performs better in terms of CPU and memory usage than a system written with JADE\@.

\subsection{Project Overview}

This project will have several milestones that need to be completed.

\begin{description}[style=nextline]
    \item [Literature Review] Background research on existing multi-agent systems primarily looking at supply chains and the technology that they use will need to be done to provide a foundation for the implementation of this project.
    \item [System Design] The multi-agent system will need to be designed so that it can be implemented in both Elixir and JADE, and distributed across multiple machines\@.
    \item [Implementation] After the system has been designed it will need to be implemented in both Elixir and JADE\@.
    \item [Analysis] Once the systems have been fully implemented they can be benchmarked and compared to see if Elixir produces a better multi-agent system.
\end{description}
