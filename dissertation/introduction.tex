\section{Introduction}

Just-in-time manufacturing is used throughout the world as the main manufacturing methodology.
The general idea behind just-in-time manufacturing is to only produce enough goods to meet demand, instead of producing the maximum amount of goods possible.
However, just-in-time manufacturing has seemingly been unable to meet customer demand when it increases very suddenly.
This project aims to use a multi-agent model to look at how situational stockpiling can be used to mitigate the issues of incorrectly predicted customer demand.

Most supply chain modelling research looks at how supply chains are inherently competition and tries to emulate this by having multiple agents competing with each other in a type of tournament.
This project instead will look at how a single manufacturing agent performs at purchasing multiple types of raw materials to assemble one type of good to demand.

This project aims to look at how product demand history can be used to determine the number of goods that will need to be produced.
Once a way to calculate the number of goods needing to be produced is implemented, the model will be run under different conditions to work out scenarios where this method does not perform well.
After these scenarios have been found, the use of stockpiling will be investigated to determine whether this is a viable countermeasure.

As mentioned previously, not a lot of supply chain modelling research is done in a non-competition environment.
This could be useful for researchers and economists who studying supply chains.

\subsection{Deliverables}

There will be two deliverables for this project.
The first will be the developed agent-based model, designed to answer the research questions below.
The second will be the analysis of the experimental data that was gathered using this model.

\subsection{Research Questions}\label{sec:research_questions}

This project seeks build a model that can answer the following research questions:

\begin{enumerate}
    \item To what extent should demand history be used for calculating the amount of goods needing to be produced.
    \item Under which circumstances is it better to stockpile raw goods and finished products.
\end{enumerate}

The first question aims to see whether economic conditions exist such that the product demand history cannot be used to accurately calculate the number of goods that would need to be produced.
The second question tries to see whether a system of predictive stockpiling can be used to mitigate the conditions found by the first question.

\subsection{Project Overview}

This project will have several milestones that need to be completed.

\begin{description}[style=nextline]
    \item [Literature Review] Background research on existing multi-agent models of supply chains and the technology that they use will need to be completed to provide a foundation for the implementation of this project.
    \item [Model Design] The multi-agent model will need to be designed in order to take full advantage of the technology chosen to implement this project it.
    \item [Model Implementation] After the model has been designed it will need to be implemented in a way that allows different simulation scenarios to be easily configured.
    \item [Data Gathering] Data will be gathered using the model in order to answer the research questions posed above.
    \item [Analysis] This data will be analysed in order to see to what extent it answers this project's research questions.
\end{description}
