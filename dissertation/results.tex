\section{Results}

This section contains the results and analysis of the 5 experiments that were described in the previous section.
The aim of this section is to answer the research questions that were proposed in the introduction of this dissertation.
The results from running the 5 experiments will be used to determine whether the multi-agent system implemented in Elixir performs better than an identical system written used JADE\@.
The source code for both systems will then be compared to see whether the Elixir system is short and potentially easier to maintain that the JADE version.

The hardware that these benchmarks were run on was an AMD Ryzen 7 3700X processor, running at a base clock speed of 3.6 Ghz and 16 logical processors.
The hardware had 32 GB of DDR4 memory, running at 3533 Mhz.
As the experiments were run under Docker, the results may differ if the experiments were to be run directly on equivalent hardware.
Docker was configured to use all the available CPU cores and to use a maximum of 25 GB of memory.

The host machine for these experiments was running Windows 10 version 21H1 build 19043.1348.
The version of Docker used to run these experiments was Docker Desktop 4.1.1.
The Elixir system was built using the Elixir version 1.12.3 Docker image.
The JADE system was built using the Maven version 3.8.3 OpenJDK 8 image.
Both of these images use Ubuntu as the base operating system.

Each experiment was run 10 times to ensure results were reproducible.
Before each experiment was run, a warmup run was conducted.
This was to ensure that the Docker image was built and that any containers that were required had been instantiated.

\subsection{CPU Usage}

The first benchmark taken was the CPU utilisation of the Manufacturer agent for each experiment.
A higher CPU utilisation means that the CPU is spending less time being idle and shows the proportion of execution time the program takes.
The measurements shown here are the average CPU utilisation over 10 runs of each experiment.

\Cref{fig:mean_cpu} shows the average CPU utilisation for the Elixir and JADE systems in each experiment.
The error bars display the standard deviation from the mean value.
These results show two interesting characteristics.
The first is that Elixir uses close to 100\% CPU utilisation on average for most experiments, while JADE uses less than 20\% for all experiments.
The second is that as the complexity of the experiments increases, both systems use less CPU utilisation.

\begin{figure}[ht]
    \centering
    \includegraphics[width=\textwidth]{mean_cpu.png}
    \caption{The mean CPU utilisation for each experiment}\label{fig:mean_cpu}
\end{figure}

As all the experiments are running under Docker it can be difficult to exactly work out why the systems are behaving in a certain way.
However, educated conjectures can be made as to the cause of these behaviours.

The first interesting characteristic could potentially be caused by the difference in threading models between Elixir and JADE\@.
Elixir by default uses as many scheduler threads as available CPU cores, for these experiments that would be 16 scheduler threads.
Any process that exists in the Elixir system can be scheduled on any of the available scheduler threads.
By contrast, JADE allocates a single dedicated thread to each agent and that agent will only ever execute on that thread.
While the JVM (Java Virtual Machine) allocates some threads for itself, the majority of work being done is happening on three threads in the JADE system.
If we assume that three threads are continuously running on a 16 machine, we can expect the CPU utilisation to be \(\frac{3}{16} \times 100 = 18.75\% \).
This is close to the actual average CPU utilisation of the JADE system.

\begin{figure}[ht]
    \centering
    \includegraphics[width=\textwidth]{cpu_per_round.png}
    \caption{The CPU utilisation per round for run 1 of experiment 5}\label{fig:cpu_per_round}
\end{figure}

\Cref{fig:cpu_per_round} shows the CPU utilisation per round in the first run of the fifth experiment.
We can see that the Elixir system spends almost the entire time at a high CPU utilisation of over 75\% with only a few spikes that are lower.
The JADE system starts with a higher than average CPU utilisation, peaking at 31.25\% utilisation.
If we assume that during the start of the benchmark there will be more load on the general JVM threads, this would explain why the system never comes close to this value of utilisation later on.

The second characteristic where the CPU utilisation decreases with complexity seems likely to be caused by more cross-machine messages being sent.
While this is difficult to verify without looking at the low-level implementation details of how messages are sent in JADE and Elixir, it seems likely that at some point when sending a message to another machine they would need to ask the OS kernel to send a network message.
As the more complex experiments increase the number of agent machines in the network, more messages between machines will be sent.
Since the systems are running in sparse Ubuntu Docker images, this context switching to kernel space to make network requests can have a measurable impact on the CPU utilisation of the system itself.

Looking at only the metric of CPU utilisation in isolation, it would seem as though JADE is better than Elixir as JADE uses few system resources.
However, it can also be interpreted that a low CPU utilisation indicates that the runtime is not using all of the system resources as effectively as it could be.
This result will need to be considered with the other runtime metrics to form a full opinion.


\subsection{Memory}\label{sec:memory}

The second benchmark taken was the average memory usage of the Manufacturer agent for each experiment.
Higher memory usage means that the multi-agent system would need more expensive hardware to run.

\Cref{fig:mean_memory} shows the average memory for the Elixir and JADE systems in each of the experiments.
The error bars indicate the standard deviation of the measurements.

\begin{figure}[ht]
    \centering
    \includegraphics[width=\textwidth]{mean_memory.png}
    \caption{The mean memory usage for each experiment}\label{fig:mean_memory}
\end{figure}

From these results, we can see that the Elixir system uses approximately the same amount of memory regardless of the complexity of the experiment.
The average memory usage for any of the experiments was always between 64 and 65 megabytes with a consistent standard deviation of around 4.15.

However, the JADE system uses more memory as the experiments become more complex, starting with a mean memory usage of 173 megabytes in the first experiment and ending up with a mean memory usage of 254 megabytes in the final experiment.
It can also be seen that the standard deviation increases as the experiments become more complex.
This shows that JADE does not consistently use the same amount of memory but rather frequently has dips or increases in memory usage.
During the first experiment, there was a standard deviation of 74 while in the final experiment there was a standard deviation of 160.

\Cref{fig:memory_per_round} shows the memory usage each round for the first run of the final experiment.
We can see that the Elixir system has only very slight deviations in terms of memory usage.
However, in this chart, we can see where the large standard deviation comes from in the JADE system.

We can see that the memory in the JADE system steadily rises before rapidly being freed.
This is a result of the Java garbage collector running.
We can note that after every two garbage collections the limit of memory that is in use before another garbage collection is called increases.
We can also see that as the experiment runs, garbage collections are called at a more frequent interval.

\begin{figure}[ht]
    \centering
    \includegraphics[width=\textwidth]{memory_per_round.png}
    \caption{The memory usage per round for run 1 of experiment 5}\label{fig:memory_per_round}
\end{figure}

These differences in memory usage can likely be explained by the differing garbage collection strategies in Elixir and JADE, as well as some of the architectural design decisions.

While both Java and Elixir use a generational garbage collector, the Java garbage collector runs globally while the Elixir garbage collector runs on a per-process level.
This means that in Java all the garbage memory will be cleared at once whereas in Elixir only the processes that produce a lot of memory garbage will be required to run their garbage collector.
This along with the use of many more scheduler threads means that the Elixir system can have certain processes currently blocked by garbage collection, however other processes can still run and act on incoming messages.

The Elixir system makes great use of ETS to store information that needs to be accessed by any of the layers in an agent.
This means that there will only ever be one copy of an element in the knowledge base and the element would only need to be cleared if it were explicitly deleted.
While a globally accessible knowledge base would be possible in JADE, this style of coding is discouraged in Java and would require the use of mutex locks to ensure thread safety.
The approach taken in the JADE system was to pass around serialisable objects that store the state of the agent.
However, this means that these objects must be cloned before they are sent to another layer and that when they are filtered during various operations in the Behaviour layer, they must be collected back into new collections.
All of these operations would produce memory garbage.

From these results, we can see that Elixir practically always uses less memory than a system using JADE\@.
While it could be possible to reduce some of the memory allocations in the JADE system, this could require dealing with mutex lock contention and other thread-safety issues which were not required in Elixir.


\subsection{Time}\label{sec:time}

The third benchmark was the mean total time for all of the 220 rounds to finish for each of the experiments.
This measurement shows how quickly the system can make decisions and execute other steps such as sending messages.

\Cref{fig:mean_time_diff} shows the average time that was required for all the experiments to finish.
The error bars show the standard deviation from the mean value.
It is immediately apparent that the JADE system is significantly slower than the Elixir system.
All of the experiments run in under 1 second for the Elixir system with most running under half a second.
The simplest experiment took on average 3 seconds for the JADE system and the most complex experiment took over 12 seconds on average to complete.

\begin{figure}[ht]
    \centering
    \includegraphics[width=\textwidth]{mean_time_diff.png}
    \caption{The mean time taken for each experiment}\label{fig:mean_time_diff}
\end{figure}

Not plotted is the average time between each round.
As each experiment uses the same number of rounds, the average time per round can be found by dividing the total time by 220.

\Cref{fig:time_per_round} shows the time taken between each round for a specific run of experiment 5.
In this chart, we can see from the trendline that the Elixir system has a fairly consistent time per round.
Interestingly, we can see that the JADE system starts slow but speeds up towards the middle of the experiment, before slowing down a bit again near the end.
This is likely caused by Java's JIT compiler which is still optimising the program for the first rounds of the experiment.

\begin{figure}[ht]
    \centering
    \includegraphics[width=\textwidth]{time_per_round.png}
    \caption{The time taken between each round for the first run of experiment 5}\label{fig:time_per_round}
\end{figure}

From these results, we can easily see that Elixir is faster than a system implemented in JADE\@.
This is an unexpected result as it was expected that the maturity of the Java platform would give JADE an advantage in terms of execution time.
Further investigation would be required to work out what is causing the JADE system to perform so slowly.
Whether it is caused by a potential bad design in the multi-agent system or whether the JADE framework code is unoptimised.


\subsection{Lines of Code}

The final metric that was measured was the total numbers of lines of code required in order to implement the multi-agent system.
This can be seen as a measure of how maintainable the code base of such a system would be.
As the fewer lines of code that are required, means that there are less lines of code that need to be checked for bugs.

This is an interesting comparison as Elixir and Java use different programming paradigms.
Java is an object-oriented programming language while Elixir is a functional programming language.
It is expected that Elixir would require less lines of code, however, potentially more functionality may need to be implemented as Elixir lacks a multi-agent framework like JADE\@.

\Cref{tab:elixir_loc} shows the total lines of code that was used to implement the Elixir system, as reported by cloc.
We can see that a total of 1107 lines of source code were used to implement the entire Elixir system.

\begin{table}[h]
    \centering
    \begin{tabular}{lrrrr}
        \toprule
        Language & files & blank & comment & code\\
        \midrule
        Elixir & 19 & 279 & 73 & 1107\\
        \midrule
        SUM\@: & 19 & 279 & 73 & 1107\\
        \bottomrule
    \end{tabular}
    \caption{Lines of code for Elixir system, reported by cloc}\label{tab:elixir_loc}
\end{table}

Table X shows the total lines of code for JADE\@.

Add Table X here.


\subsection{Summary}

From the results gathered by running all five experiments, we can see that Elixir produces a system that is significantly more memory efficient and runs significantly faster.
The downside is that it utilises practically all of the CPU resources available.

The high CPU utilisation may be a product of the fact that in these experiments after one round finishes, the next round is immediately started.
This provides the system with no downtime, resulting in all of the 16 scheduler threads being constantly active.
If these experiments were modelled more closely to the round structure of TAC SCM where each round always takes 15 seconds, it is likely that the Elixir CPU utilisation may be closer to the JADE values.
A multi-agent system where each round always starts immediately after the last round is not very representative of a real-world system which would be closer to a real-time system with periods of downtime.

The consistently low memory usage of the Elixir system would be a benefit for using the language in a real-system.
It would mean that an agent in a supply chain could be hosted on cheaper hardware and would not require a larger memory capacity just to deal with spikes in memory usage.

The low execution time for the Elixir system would also be a benefit in a real-world system, as it indicates that the agent would be able to make decisions faster.
Being able to make a decision faster means that a business plan can be put in place or adjusted quickly while the information the agent received is still current and accurate.

The Elixir system was also found to be shorter in terms of lines of code than the JADE system.
This shows that using a dedicated framework for agent-based programming may potentially not help in implementing a concise system.
As systems written in Elixir are shorter it would allow them to be maintained with greater ease.
This would help in a real-world system where the behaviour of an agent may need to be modified after it has been deployed.
