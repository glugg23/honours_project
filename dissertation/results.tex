\section{Results}

This section contains the results and analysis of the 5 experiments that were described in the previous section.
The aim of this section is to answer the research questions that were proposed in the introduction of this dissertation.
The results from running the 5 experiments will be used to determine whether the multi-agent system implemented in Elixir performs better than an identical system written used JADE\@.
The source code for both systems will then be compared to see whether the Elixir system is short and potentially easier to maintain that the JADE version.

The hardware that these benchmarks were run on was an AMD Ryzen 7 3700X processor, running at a base clock speed of 3.6 Ghz and 16 logical processors.
The hardware had 32 GB of DDR4 memory, running at 3533 Mhz.
As the experiments were run under Docker, the results may differ if the experiments were to be run directly on equivalent hardware.
Docker was configured to use all the available CPU cores and to use a maximum of 25 GB of memory.

The host machine for these experiments was running Windows 10 version 21H1 build 19043.1348.
The version of Docker used to run these experiments was Docker Desktop 4.1.1.
The Elixir system was built using the Elixir version 1.12.3 Docker image.
The JADE system was built using the Maven version 3.8.3 OpenJDK 8 image.
Both of these images use Ubuntu as the base operating system.

Each experiment was run 10 times to ensure results were reproducible.
Before each experiment was run, a warmup run was conducted.
This was to ensure that the Docker image was built and that any containers that were required had been instantiated.

\subsection{CPU Usage}

Display results for average CPU usage per experiment.


\subsection{Memory}

Display results for average memory usage per experiment.


\subsection{Time}\label{sec:time}

The third benchmark was the mean total time for all of the 220 rounds to finish for each of the experiments.
This measurement shows how quickly the system can make decisions and execute other steps such as sending messages.

\Cref{fig:mean_time_diff} shows the average time that was required for all the experiments to finish.
The error bars show the standard deviation from the mean value.
It is immediately apparent that the JADE system is significantly slower than the Elixir system.
All of the experiments run in under 1 second for the Elixir system with most running under half a second.
The simplest experiment took on average 3 seconds for the JADE system and the most complex experiment took over 12 seconds on average to complete.

\begin{figure}[h]
    \centering
    \includegraphics[width=\textwidth]{mean_time_diff.png}
    \caption{The mean time taken for each experiment}\label{fig:mean_time_diff}
\end{figure}

Not plotted is the average time between each round.
As each experiment uses the same number of rounds, the average time per round can be found by dividing the total time by 220.

\Cref{fig:time_per_round} shows the time taken between each round for a specific run of experiment 5.
In this chart, we can see from the trendline that the Elixir system has a fairly consistent time per round.
Interestingly, we can see that the JADE system starts slow but speeds up towards the middle of the experiment, before slowing down a bit again near the end.
This is likely caused by Java's JIT compiler which is still optimising the program for the first rounds of the experiment.

\begin{figure}[ht]
    \centering
    \includegraphics[width=\textwidth]{time_per_round.png}
    \caption{The time taken between each round for the first run of experiment 5}\label{fig:time_per_round}
\end{figure}

From these results, we can easily see that Elixir is faster than a system implemented in JADE\@.
This is an unexpected result as it was expected that the maturity of the Java platform would give JADE an advantage in terms of execution time.
Further investigation would be required to work out what is causing the JADE system to perform so slowly.
Whether it is caused by a potential bad design in the multi-agent system or whether the JADE framework code is unoptimised.


\subsection{Lines of Code}

Display results for lines of code in Elixir and JADE systems.


\subsection{Summary}

From the results gathered by running all five experiments, we can see that Elixir produces a system that is significantly more memory efficient and runs significantly faster.
The downside is that it utilises practically all of the CPU resources available.

The high CPU utilisation may be a product of the fact that in these experiments after one round finishes, the next round is immediately started.
This provides the system with no downtime, resulting in all of the 16 scheduler threads being constantly active.
If these experiments were modelled more closely to the round structure of TAC SCM where each round always takes 15 seconds, it is likely that the Elixir CPU utilisation may be closer to the JADE values.
A multi-agent system where each round always starts immediately after the last round is not very representative of a real-world system which would be closer to a real-time system with periods of downtime.

The consistently low memory usage of the Elixir system would be a benefit for using the language in a real-system.
It would mean that an agent in a supply chain could be hosted on cheaper hardware and would not require a larger memory capacity just to deal with spikes in memory usage.

The low execution time for the Elixir system would also be a benefit in a real-world system, as it indicates that the agent would be able to make decisions faster.
Being able to make a decision faster means that a business plan can be put in place or adjusted quickly while the information the agent received is still current and accurate.

The Elixir system was also found to be shorter in terms of lines of code than the JADE system.
This shows that using a dedicated framework for agent-based programming may potentially not help in implementing a concise system.
As systems written in Elixir are shorter it would allow them to be maintained with greater ease.
This would help in a real-world system where the behaviour of an agent may need to be modified after it has been deployed.
