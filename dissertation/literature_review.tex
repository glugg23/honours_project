\section{Literature Review}

Introduction here.

\subsection{Modelling Supply Chains}

\subsection{Agent-based Design}

\subsection{Technologies for Agent-based Modelling}

\subsubsection{Erlang}

The Erlang\footnote{\url{https://www.erlang.org/}} programming language was introduced by \citeA{armstrong1990erlang} as an experimental new programming language to be used for programming telephony systems.
Erlang is designed around using concurrent processes with no shared memory, which communicate using asynchronous message passing.
All of this is built into the language definition.~\cite{armstrong2007history}
The fact that this behaviour is implemented on a language basis and not as a third-party library is beneficial as it allows us to implement the basic Actor principle of message passing entirely using Erlang processes.

\citeA{varela2004modelling} describes a method for implementing agents using multiple Erlang processes to represent individual components of an agent.
These processes communicate with one another to execute overall actions for the agent, and one or more processes are dedicated to communicating with other agents in the system.
They also describe the notion of a supervisor tree, which is a tree-like hierarchy of agents where each agent monitors the state of any agents below it.
This allows fault tolerance as long as the supervisor stores the initial state of all the agents it monitors.

\citeA{di2005using} evaluated Erlang on three metrics: Agent Model Compliance, Support for Rationality and Support for Distribution.
They found that while Erlang fully met the criteria for distribution support, it did not fully support agent behaviour composition and extension, which would be required for agent model compliance.
They also found that Erlang lacked logical inference support to provide rationality to agents.
They propose the eXAT\footnote{\textbf{e}rlang e\textbf{X}perimental \textbf{A}gent \textbf{T}ool} platform as a suitable library for fixing these issues.~\cite{di2003exat}
However, \citeA{peregud2012implementing} found that the eXAT platform was cumbersome to write models with, was practically discontinued since 2005 and has no community support and no production systems using it.

\citeA{santana2017interscsimulator} have used Erlang to model a large scale traffic simulation using millions of agents.
They note that the parallelism provided by Erlang processes and the ability to run Erlang code distributed on multiple machines simultaneously are the main factors for the performance they achieved.
However, they mention that the drawbacks of Erlang are a lack of process synchronisation mechanisms and a general lack of IDE\footnote{Integrated Development Environment} tooling.

While Erlang has been used for implementing other types of scientific modelling, it has yet to be used for modelling supply chains.

\subsubsection{Elixir}

The Elixir\footnote{\url{https://elixir-lang.org/}} programming language can be considered to be a more modern version of Erlang.
Elixir compiles to the same byte code that the Erlang runtime uses.
This allows Elixir code to call code written in Erlang with no runtime overhead.
Elixir uses a syntax similar to Ruby, which is more modern than the Prolog inspired syntax of Erlang.~\cite{loder2016erlang}

As Elixir is a fairly new programming language it has yet to be used for any type of modelling research.
\citeA{fedrecheski2016elixir} compared Elixir with Java for use in Internet of Things software.
They found that while the Java implementation used 5\% to 15\% less CPU load, the Elixir version was shorter in terms of lines of code, consumed significantly less memory, and handled HTTP responses better under heavy load.

While Elixir has not yet found much use in the scientific community, it has been used for web development and real-time applications.
Discord had been an early adopter of Elixir for its messaging platform and was able to scale its application to handle five million concurrent users using it.~\cite{vishnevskiy2017discord}
\citeA{pinterest2017introducing} have used Elixir to implement a notification system.
They found that the Elixir version had ten times fewer lines of code than the original Java version, and ran faster as well.

For this project I have chosen to use Elixir for implementing my supply chain model.
I believe that the benefits of using a language that can leverage the features of Erlang while simultaneously having a more modern syntax, outweighs the potential issues of using a language that has not yet been used for research purposes.

\subsubsection{JADE}

JADE (\textbf{J}ava \textbf{A}gent \textbf{DE}velopment Framework)\footnote{\url{https://jade.tilab.com/}} is a Java framework for developing agent-based applications, and was one of the first FIPA specification compliant frameworks.~\cite{bellifemine1999jade}
JADE provides a platform for executing agents via composable behaviours and allowing both local and distributed agents to communicate with each other.
One optimisation that JADE provides is that FIPA ACL messages are transported between JADE agents on the same platform as Java objects, instead of needing to be serialised into a string format.
For agents on different platforms the message is converted as required but this provides a performance boost for local agents.

\citeA{bergenti2020first} note that in the 20 years that the JADE framework has been available, it has been extensively used in academia for software agent research as well as being used to introduce students to agent-oriented programming.
However, \shortciteA{bergenti2020first} also mention that students using JADE struggle with the complexity of the framework, if they do not have any prior knowledge of software agents.
They propose Jadescript as a language to provide a dedicated syntax for constructing and using JADE agents.

JADE has been used by \citeA{podobnik2006crocodileagent} for implementing an agent to compete in TAC SCM\@.
As they were using an IKB agent model, they mention that a benefit of JADE was that they could separate each layer onto a different computer.
While this results in a more complex system due to requiring lots of intercommunication, they still managed to reach the semi-finals of the competition once and the quarter-finals twice.~\cite{collins2009flexible}

\subsubsection{NetLogo}

NetLogo\footnote{\url{https://ccl.northwestern.edu/netlogo/}} is a multi-agent programming language which provides an environment for modelling complex systems over time
It has been designed for use in both education and research, and since it has been written using Java it can be run on any major operating system.
NetLogo uses agents called ``turtles'' to move over and interact with a grid of ``patches''.
Both of these can be programmed to have certain behaviours.
NetLogo uses a primarily graphical interface to make it easy to view and modify the simulation while it is running.~\cite{tisue2004netlogo}

While NetLogo is popular in general for agent-based modelling, only a few SCM models have used it.
\citeA{arvitrida2015competition} used NetLogo to model the effect of competition and collaboration on supply chain performance.
They selected NetLogo for being relatively simple but still providing all the features that they needed for this model.

\subsubsection{SARL}

SARL\footnote{\url{http://www.sarl.io/}} is a general-purpose agent-based programming language that aims to be both platform and architecturally agnostic.
SARL provides native support for agent-oriented first-class abstractions but does not force the programmer to use them in any specific way.
The reason why SARL is designed to be agnostic is the belief that a more general and less research focused or theoretical approach is required for agent-oriented programming to make a more significant impact on mainstream software engineering.
SARL compiles to Java byte code and can be run on the Janus platform, although other platforms can be used as well.~\cite{rodriguez2014sarl}

As SARL is a relatively new agent-oriented programming language, it has yet to be used for SCM and has also not seen much use for general agent-based modelling yet.
