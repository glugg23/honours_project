\subsection{Experiments}

In order to be able to compare the two systems, a set of experiments need to be defined so that it is possible to see how the systems perform in different scenarios.
It is important to use multiple scenarios as this would find potential bottlenecks that might not be visible in only a simple experiment.
This is also relevant as in the real world a multi-agent system, especially one for supply chains would need to be able to handle varied and dynamic environments.
For simplicity, all the experiments run will be static.
This means that no new agents will intentionally join or leave the simulation while it is running.

The variables that will be adjusted in each experiment are, the number of producer agents that are available, the base goods that are available to be produced and the recipes for what PCs can be manufactured.
This results in experiments that increase the distributed complexity of the simulation.
This direction of experimentation was chosen over making the behaviour of each agent more complex, as it scales better and could provide more insightful results for real-world systems.
This is because if only the complexity of agent behaviours increase they may end up being over-optimised for a single scenario and provide less general results.
Ideally, each scenario should be run with a set of simple and complex behaviours but this is not possible in the time provided.

The first experiment is a simple experiment where there is only one Producer agent which produces a single component type.
This good type can be directly converted into a manufactured PC at a one to one rate.
This is a very simple experiment as there are no decisions that need to be made about what goods should be purchased and what types of PCs need to be manufactured.
The performance in this experiment would be considered a best-case scenario.

The second experiment is similar to the first experiment but that there are now two Producer agents that produce this single component type.
This means that while the act of manufacturing PCs is still simple given that there is only one recipe to do so, the Manufacturer agent now needs to make decisions about where it should buy components from.
This is now a slightly more realistic example as in real supply chains there will be multiple producers of the same kind of component.

The third experiment changes the recipe for the PC production to use two different component types.
There are two Producer agents in this scenario, one for each component type.
Now instead of having to decide where to buy a single component, the Manufacturer agent needs to handle buying different components from different places.
This is again coming closer to a realistic scenario as the manufacturing of computers requires multiple different components.

The fourth experiment combines aspects of the second and third experiment by using a two-component PC recipe where each component has two Producer agents that produce it.
This now requires the Manufacturer agent to both handle multiple component types as well as being able to procure them from different agents.

The final experiment is based on the scenario from TAC SCM but has been simplified to avoid adding extra development effort.
The TAC SCM scenario has multiple producers for CPUs, motherboards, memory and hard drives, and each producer can create two types of components.
Additionally, CPU types need to be paired up with specific motherboard types in order to be compatible.
This level of complexity means that TAC SCM defines 16 valid recipes for producing various types of computers.
Recreating this scenario is a little out of scope for this project as it would require implementing the ability for Producer agents to produce different types of components, which is a feature that would only be required for this benchmark.
Instead, the scenario will be that there are two recipes for producing PCs, one recipe for a ``slow'' computer and another recipe for a ``fast'' computer.
These recipes would require 4 different components where two of them are unique to that recipe.
This experiment will use 8 Producer agents, two for each of the common components and then one Producer agent per unique component.
