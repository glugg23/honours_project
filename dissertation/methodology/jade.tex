\subsection{JADE}

This section looks at how the JADE system has been implemented.
It compares the JADE system to the Elixir system to see what aspects have been kept the same and what needed to be changed.

The JADE system has been written using Java 8.
Although this is a fairly out of date Java version, it is the last version where the JADE framework can be compiled on.
While this is not an issue for this project as Maven is being used as a build tool that uses a jar file to provide access to JADE libraries, some projects might compile JADE themselves and so it would make sense to use a version of Java that can still compile JADE\@.

Each agent layer in the JADE system has been implemented as a separate agent.
This maps more closely to how the system has been designed in Elixir.
Each internal agent communicates with the other by sending messages.
Although a single unified agent could be used, the choice of using multiple agents was made to more rigorously separate the layers in the IKB framework.

The configuration of the JADE system has been done programmatically instead of by passing flags to the JADE boot class.
This ensures that the same configuration is used for all experiments.
Which type of layers need to be started is also controlled programmatically, based on the value of a system environment variable.

\subsubsection{Information Layer}

Compared to the Information layer in the Elixir system, the Information layer in the JADE system is a lot simpler.

When the agent for the Information layer starts, it requests a list of all agents from the Knowledge layer and uses this for the information filter.
Once it receives a message it forwards the message to the Knowledge layer if the sender is not in the information filter.

A new behaviour class was implemented which acts similar to the GenServer Erlang behaviour.
This abstract behaviour class is similar to the ``CyclicBehaviour'' class provided by JADE\@.
However, the ``action'' method has been overridden to perform a blocking receive for a message.
This message is then passed to a ``handle'' method which is defined as being abstract and needs to be implemented by any agent using this behaviour.

The Information layer uses this behaviour class to wait on messages and overrides the ``handle'' method to do any message filtering.

\subsubsection{Knowledge Layer}

The Knowledge layer acts as the knowledge base in the system.
Since JADE does not have an equivalent to ETS, the Knowledge layer agent is instead used for storing all the knowledge.

Similar to the macros used in Elixir for reducing code duplication, a single abstract Knowledge agent has been defined which implements common functionality.
Each instance of the Knowledge agent also uses a ``KnowledgeBehaviour'' class which extends the ``GenServerBehaviour'' class to provide common but specific message responses for the Knowledge layer.

The abstract Knowledge agent first loads the config file for the experiment as well as loading an XML file for the information about every agent in the multi-agent system.
The configuration for each experiment is saved in a properties file as key-value string pairs.
Unlike Elixir where types such as arrays and maps can be used directly in the config file, Java only supports string data unless the config file is a serialised object.
Therefore, some amount of code is required to parse the config values in the properties file.

Notably, the JADE system does not attempt to dynamically connect to other agents in the network which is what the Elixir system does.
JADE supports this functionality under the name of federations where multiple Directory Facilitator (df) agents can be connected together.
Once multiple df agents have been connected, any agent in the same container can query all the connected df agents to find an agent providing a certain service.
In practice, this functionality was found to be inconsistent at working correctly and frequently resulted in race conditions where one or more agents would fail to join the federation.
From reading the JADE mailing lists, \citeA{caire2009problem} says ``that typically DF federations \dots{} are performed by the administrator `by hand'\,''.
As this is not possible when running the system under Docker, federations were not used in this system.

Instead of using federations, an XML file was provided for each experiment which contains all the connection information for each agent in the system.
This XML file is a serialised HashMap object which is loaded into the system by the Knowledge layer.

Similar to the Elixir system, when the JADE system receives a message from the Information layer it is placed into a HashMap called the inbox.

\subsubsection{Behaviour Layer}

The Behaviour layer in the JADE system has also been implemented as a finite state machine, as was done in the Elixir system.
This finite state machine uses the ``FSMBehaviour'' class from JADE and defines states for each element of the agent's behaviour.
States for this class are defined as classes that extend an existing JADE behaviour class.
For the state machines that have been defined in this project, all the states extend the ``OneShotBehaviour'' class.

Due to the states being added via polymorphic classes, any references to the agent that the behaviour belongs to will always be of the superclass ``Agent''.
This means that any class member variables will not be accessible in the behaviour without casting the Agent instance to the specialised type.
To avoid this issue, each state class has a constructor which takes the specialised agent class and assigns it to a member variable in the state class.
This variable is then used whenever a reference to the agent is required.

At the start of the round, the Behaviour layer requests a copy of the agent state from the Knowledge layer.
As Java does not have a globally accessible place to store data such as ETS, this was the approach chosen to synchronise information between layers.
The downside of this approach is that the Behaviour layer is only sent a snapshot of the current knowledge base.
This means that if a new message were to be received the behaviour layer would be unaware of it.
To fix this issue another command was to the Knowledge layer which allows the Behaviour layer to request a new copy of just the inbox.
At the end of the round, the Behaviour layer sends the modified agent state back to the Knowledge layer.

While JADE allows agents to send Java objects in messages if they are on the same machine, this is not possible between remote agents.
Due to needing structured data in the form of objects when agents make requests, a manual way to encode request objects in a string format has been implemented.
This is only possible because it is known that all the agents are running the same version of JADE and would likely not be an adequate solution for a real-world system.
For a real-world, system a text-based representation such as JSON or XML could be used.
