\subsection{JADE}

Introduction to JADE system.

Using Java 8 as this is the last long term support version where JADE can be compiled, although due to the nature of Java bytecode a later version could be used.

Each layer is a separate agent instance.

At start-up the specific layers are started depending on the system environment.

\subsubsection{Information Layer}

Compared to the Information layer in the Elixir system, the Information layer in the JADE system is a lot simpler.

When the agent for the Information layer starts, it requests a list of all agents from the Knowledge layer and uses this for the information filter.
Once it receives a message it forwards the message to the Knowledge layer if the sender is not in the information filter.

A new behaviour class was implemented which acts similar to the GenServer Erlang behaviour.
This abstract behaviour class is similar to the ``CyclicBehaviour'' class provided by JADE\@.
However, the ``action'' method has been overridden to perform a blocking receive for a message.
This message is then passed to a ``handle'' method which is defined as being abstract and needs to be implemented by any agent using this behaviour.

The Information layer uses this behaviour class to wait on messages and overrides the ``handle'' method to do any message filtering.

\subsubsection{Knowledge Layer}

Acts as the knowledge base in the system.

There is an abstract Knowledge agent which the others extend.
This agent defines common behaviour like the Elixir macros.

This class loads config values from a properties file.

Write about attempt to form agent federation and why this did not work.

Write about de-serialising agent information from XML file and why JADE custom collections made serialising JADE objects more difficult.

\subsubsection{Behaviour Layer}

Implemented using JADE's finite state machine behaviour.

Explain the behaviour of the different agent types.
