\subsection{Elixir}

Introduction to Elixir system.

This system makes heavy use of the standard Erlang OTP behaviour\\ \verb|gen_server| (henceforth referred to as GenServer, following standard Elixir naming conventions).
A GenServer is implemented as a Client-Server relationship where a public API is used to interact with user-defined callback functions.
The core behaviour of a GenServer is to wait for a message to be received, then pass that message to the appropriate callback function before looping back to wait for the next message.
This behaviour fits neatly in line with the general behaviour of agents, where the reaction to a message can be implemented as callback functions.

The split Client-Server relationship of a GenServer can be used to define a common interface that all server implementations need to adhere to.
By defining all client-facing methods in a separate module and implementing separate callback functions in different modules, we can guarantee that all particular implementations of an agent can be interacted with in the same manner.
Callback functions can also be implemented without requiring a public interface method.
This allows the GenServer to handle arbitrary messages that have been sent to it.
This functionality has been predominantly used in this system as many messages are being sent between different layers and different network nodes.

While Elixir allows the sending of any data type between different processes, it lacks a defined message type such as the ACLMessage class in JADE\@.
For this system, a message type has been defined by using the struct feature of Elixir.
This means we can define a struct that is similar to a map where only certain keys are allowed.
Certain keys can also be enforced meaning that attempting to initialise a struct with these keys missing would be an error.
The message struct that was defined for this system is based on a subset of fields that are used in FIPA ACL messages.
The message struct contains keys for the performative, message sender, message receiver, whom to reply to, the content of the message and the message ID\@.
Helper functions for replying to, forwarding and sending the message structs have also been implemented.
Not all the messages in this system use this message format.
Where appropriate, such as the callback functions for GenServers use simpler messages such as tuples of primitives.

Instead of creating a separate project for each agent type, a single project is used for all agents.
This is beneficial as all the agent types share a common information layer implementation and reuse common knowledge layer responses.
The type of the agent is set at the start up of the application by reading the system environment variables and checking the value of the \verb|AGENT_TYPE| variable.
This variable indicates which knowledge and behaviour layer implementations should be started.
If this variable is not set or is set to an unrecognised value, then the agent application will exit with an error.

Config loaded in via Elixir script file.

\subsubsection{Information Layer}

Handles inter-node connections by listening for nodeup events.

Library ``libcluster'' is used to automatically connect to other agents on the same network.

Builds the information filter for the agent.

Stores the node information in ETS, explain what ETS is.

Forwards any messages that pass through the information filter to the Knowledge layer.

\subsubsection{Knowledge Layer}

Loads config values into knowledge base.
Knowledge base is also stored in ETS\@.

Is responsible for updating Knowledge base as messages are received.

Each agent uses a separate Knowledge layer implementation.
Common functionality is included via the use of macros.

\subsubsection{Behaviour Layer}

The proactive and reactive behaviour of the agent.

Is implemented as a finite state machine using the library ``GenStateMachine'' which is a wrapper class for the Erlang module \verb|gen_statem|.

Explain the behaviour of the different agent types.
