\subsection{Elixir}

Introduction to Elixir system.

This system makes heavy use of the standard Erlang OTP behaviour\\ \verb|gen_server| (henceforth referred to as GenServer, following standard Elixir naming conventions).
A GenServer is implemented as a Client-Server relationship where a public API is used to interact with user-defined callback functions.
The core behaviour of a GenServer is to wait for a message to be received, then pass that message to the appropriate callback function before looping back to wait for the next message.
This behaviour fits neatly in line with the general behaviour of agents, where the reaction to a message can be implemented as callback functions.

The split Client-Server relationship of a GenServer can be used to define a common interface that all server implementations need to adhere to.
By defining all client-facing methods in a separate module and implementing separate callback functions in different modules, we can guarantee that all particular implementations of an agent can be interacted with in the same manner.
Callback functions can also be implemented without requiring a public interface method.
This allows the GenServer to handle arbitrary messages that have been sent to it.
This functionality has been predominantly used in this system as many messages are being sent between different layers and different network nodes.

Required implementing own message format.

Agent type loaded in via system environment.

Config loaded in via Elixir script file.

\subsubsection{Information Layer}

Handles inter-node connections by listening for nodeup events.

Library ``libcluster'' is used to automatically connect to other agents on the same network.

Builds the information filter for the agent.

Stores the node information in ETS, explain what ETS is.

Forwards any messages that pass through the information filter to the Knowledge layer.

\subsubsection{Knowledge Layer}

Loads config values into knowledge base.
Knowledge base is also stored in ETS\@.

Is responsible for updating Knowledge base as messages are received.

Each agent uses a separate Knowledge layer implementation.
Common functionality is included via the use of macros.

\subsubsection{Behaviour Layer}

The proactive and reactive behaviour of the agent.

Is implemented as a finite state machine using the library ``GenStateMachine'' which is a wrapper class for the Erlang module \verb|gen_statem|.

Explain the behaviour of the different agent types.
