\subsection{Elixir}

This section describes how the Elixir system has been implemented.
It looks at the general design decisions that were taken as well as the implementation of the Information, Knowledge and Behaviour layers.

This system makes heavy use of the standard Erlang OTP behaviour\\ \verb|gen_server| (henceforth referred to as GenServer, following standard Elixir naming conventions).
A GenServer is implemented as a Client-Server relationship where a public API is used to interact with user-defined callback functions.
The core behaviour of a GenServer is to wait for a message to be received, then pass that message to the appropriate callback function before looping back to wait for the next message.
This behaviour fits neatly in line with the general behaviour of agents, where the reaction to a message can be implemented as callback functions.

The split Client-Server relationship of a GenServer can be used to define a common interface that all server implementations need to adhere to.
By defining all client-facing methods in a separate module and implementing separate callback functions in different modules, we can guarantee that all particular implementations of an agent can be interacted with in the same manner.
Callback functions can also be implemented without requiring a public interface method.
This allows the GenServer to handle arbitrary messages that have been sent to it.
This functionality has been predominantly used in this system as many messages are being sent between different layers and different network nodes.

While Elixir allows the sending of any data type between different processes, it lacks a defined message type such as the ACLMessage class in JADE\@.
For this system, a message type has been defined by using the struct feature of Elixir.
This means we can define a struct that is similar to a map where only certain keys are allowed.
Certain keys can also be enforced meaning that attempting to initialise a struct with these keys missing would be an error.
The message struct that was defined for this system is based on a subset of fields that are used in FIPA ACL messages.
The message struct contains keys for the performative, message sender, message receiver, whom to reply to, the content of the message and the message ID\@.
Helper functions for replying to, forwarding and sending the message structs have also been implemented.
Not all the messages in this system use this message format.
Where appropriate, such as the callback functions for GenServers use simpler messages such as tuples of primitives.

Instead of creating a separate project for each agent type, a single project is used for all agents.
This is beneficial as all the agent types share a common information layer implementation and reuse common knowledge layer responses.
The type of the agent is set at the start up of the application by reading the system environment variables and checking the value of the \verb|AGENT_TYPE| variable.
This variable indicates which knowledge and behaviour layer implementations should be started.
If this variable is not set or is set to an unrecognised value, then the agent application will exit with an error.

The config for each agent type is loaded in as an Elixir script file, using the Config module in the Elixir standard library.
An Elixir script file is a source code file that is run by being interpreted at runtime instead of being compiled to bytecode.
By importing the Config module, macros can be used to define key-value pairs in a scoped namespace.
This is used to store the initial knowledge for each agent type.
Since this is just an Elixir source code file, any Elixir datatypes can be used.
This means that complex types such as maps or lists of values can be used without having to parse them from a markup language like XML\@.
This is useful as it unifies the language that is used for configuration with the language that is used for code.

\subsubsection{Information Layer}

The Information layer in the Elixir system is used to handle the discovery and connection between all the different nodes in the distributed system.
The library \verb|libcluster|\footnote{\url{https://github.com/bitwalker/libcluster}} is used to handle the automatic discovery of any nodes in the local network.
This library is started in the main application supervisor tree alongside the individual supervisor trees for each layer.
It is configured to use the \verb|Gossip| clustering strategy.
This strategy uses multicast UDP (User Datagram Protocol) to send information packets across the network and forms connections with any nodes that are listening for it.

The Information layer uses the Erlang module \verb|net_kernel| function\\ \verb|monitor_nodes/1| to received messages when a new connection is formed between nodes.
This message takes the form \verb|{:nodeup, node}| where node is the name of the node that was connected to.
Once the Information layer has received this message that a node has connected, it can use the name of the node in order to request information about it.

After the Information layer has requested information about a new node, it adds it to its information filter.
First it looks up if the node is of a type that should be ignored.
If it is it sets a boolean value to be true which indicates that messages from that node should always be ignored.

The information filter is stored as a table in Erlang Term Storage.
Erlang Term Storage (ETS) is a runtime provided, in-memory key-value store.
Each process can create new tables in order to store data.
ETS allows for any type to be stored directly, meaning there is no need to convert between representations.
ETS provides constant time access to data and tables can be configured to be accessible globally.
This makes it ideal for storing information that is required by all of the layers in an agent.

When a message arrives the Information layer checks whether the sender is in the information filter or not.
If the sender is not in the filter, then the message is passed on to the Knowledge layer.
Otherwise, the message is ignored.

\subsubsection{Knowledge Layer}

The Knowledge layer is responsible for loading the initial knowledge into the agent.
It reads a config file which stores all the knowledge that an agent needs to know before a simulation starts.
Then these values are inserted into a table that is stored in ETS\@.
This allows the Behaviour layer to read and update the knowledge base as well.

When the Knowledge layer receives a forwarded message from the Information layer, it gets inserted into an ETS table called the Inbox.
This allows messages to be handled in bulk by the Behaviour layer.

At the end of the simulation the Knowledge layer prints out the total amount of money that an agent has made.

Each agent uses a separate Knowledge layer implementation in order to be able to handle unique message types.
As Elixir lacks a form of class inheritance, macros are used to ensure duplicate behaviour only needs to be written once.

\subsubsection{Behaviour Layer}

The Behaviour layer defines how the agent should react to messages and what its proactive behaviour should be.

The Behaviour layer is implemented as a finite state machine using the library \verb|GenStateMachine|\footnote{\url{https://github.com/ericentin/gen_state_machine}} which is a wrapper for the Erlang module \verb|gen_statem|.
This library was chosen as using the base Erlang module in Elixir is cumbersome.
As this library does little than provide an Elixir wrapper over the Erlang functionality, it was considered acceptable to use.

Each state is defined as a function that handles an event.
An event is caused by a message being received, although this does not have to exclusively be external messages.
Most transitions are caused by internal messages.
The return value of each of these functions defines what state the state machine should move to.

The behaviour of each Behaviour layer starts when the agent receives a message saying that a new round has started.
After all the states have been finished, a message gets sent to the clock agent saying that the sender is finished.
After all 220 rounds have been finished, the agent shuts down and stops the runtime.
