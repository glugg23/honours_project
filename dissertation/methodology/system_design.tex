\subsection{System Design}

This section covers what each component is in the multi-agent system that has been implemented.
The multi-agent system that was implemented in order to benchmark Elixir and JADE is a PC manufacturing simulation where an agent tries to meet the demand of consumers by buying raw components and producing finished PCs.
This is similar to the scenario in TAC SCM but is less complex as there is only one manufacturing agent.
A description is given of each of the agent types in this multi-agent system.
An overview of what messages are sent between the agents in a typical round is also given.

\subsubsection{Agent Types}

There are 4 agent types that have been implemented for both of the multi-agent systems.
These are the minimum number of agents types to allow the benchmarks to be run.

The first agent type is the Clock agent.
This agent is not a part of the supply chain simulation, but instead keeps track of the simulation itself.
It makes sure all the agents are ready before starting the simulation.
While the simulation is running, the Clock agent keeps track of what round the simulation is currently on and which agents are already finished.

The second type of agent is the Producer agent.
This agent produces the raw goods or components that can be manufactured into finished PCs.
The Producer agent ignores any messages it receives from the consumer agent as messages from it are not relevant to the Producer agent.
It also ignores messages from other Producer agents as they would not be relevant to it.
This agent responds to messages from the Manufacturer agent about requests to buy components.
If the request is for a reasonable amount of components and the price per component is acceptable, then the Producer agent will accept the request.
There can be more than one Producer agent per experiment.

The next type of agent is the Manufacturer agent.
This agent takes requests for finished PCs from the Consumer agent and attempts to fulfil them by ordering components from Producer agents.
It will reject requests from the Consumer agent if they are at a too low price or request too many finished PCs in too short of a time frame.
This is the most complex agent type as it has to interact with both the Consumer agent and potentially multiple Producer agents.
The Manufacturer agent has been written without the assumption of competition so adding multiple Manufacturer agents would require a significant rewrite of their behaviour.
As such there will only be one Manufacturer agent in the simulation.

The final agent type is the Consumer agent.
This agent makes a number of requests to buy finished PCs from the Manufacturer agent each round.
It will offer to buy a certain number of PCs at a given price and any offers that are not accepted during the next round will be regarded as being rejected.
The Consumer agent ignores all messages from the Producer agent as it does not care about raw goods.
There will only ever be one Consumer agent as adding multiple agents would just result in there being too many orders for the Manufacturer agent to fulfil.

\subsubsection{Simulation Round Messages}

\Cref{fig:system_sequence_diagram} shows a sequence diagram for the messages that are exchanged during a typical simulation round.
Agents always react to the messages they were sent in the previous round instead of reacting to the messages they are sent in the current round.
This can be seen in \cref{fig:system_sequence_diagram} where the Manufacturer agent sends an ``accept offer'' message to the Consumer, accepting an offer made in the previous round and not the offer that the Manufacturer agent will receive later.

\begin{figure}[ht]
    \centering
    \includegraphics[width=\textwidth]{system_sequence_diagram.png}
    \caption{A sequence diagram showing an example round of the simulation}\label{fig:system_sequence_diagram}
\end{figure}

Each round is started by the Clock agent announcing the start of the round by sending a message to each of the other agents.
This causes the other agents to begin reacting to the messages they were sent in the previous round.
A summary of all the messages that are usually sent is given below.

The Consumer agent will send messages to all other agents announcing what orders it wishes to buy.
The Manufacturer agent will respond to order messages from the Consumer agent by accepting or rejecting them.
It will then make offers to specific Producer agents in order to procure components.
The Producer agents will send messages broadcasting their available stock to all other agents.
They will also respond to messages from the Manufacturer agent to accept or reject orders to buy components.

Once an agent is finished it notifies the Clock agent by sending it a message.
When the Clock agent has received messages from all agents that indicate they are finished, it moves to start the next round of the simulation.
