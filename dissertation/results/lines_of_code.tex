\subsection{Lines of Code}

The final metric that was measured was the total numbers of lines of code required in order to implement the multi-agent system.
This can be seen as a measure of how maintainable the code base of such a system would be.
As the fewer lines of code that are required, means that there are less lines of code that need to be checked for bugs.

This is an interesting comparison as Elixir and Java use different programming paradigms.
Java is an object-oriented programming language while Elixir is a functional programming language.
It is expected that Elixir would require less lines of code, however, potentially more functionality may need to be implemented as Elixir lacks a multi-agent framework like JADE\@.

\Cref{tab:elixir_loc} shows the total lines of code that was used to implement the Elixir system, as reported by cloc.
We can see that a total of 1107 lines of source code were used to implement the entire Elixir system.

\begin{table}[h]
    \centering
    \begin{tabular}{lrrrr}
        \toprule
        Language & files & blank & comment & code\\
        \midrule
        Elixir & 19 & 279 & 73 & 1107\\
        \midrule
        SUM\@: & 19 & 279 & 73 & 1107\\
        \bottomrule
    \end{tabular}
    \caption{Lines of code for Elixir system, reported by cloc}\label{tab:elixir_loc}
\end{table}

Table X shows the total lines of code for JADE\@.

Add Table X here.
