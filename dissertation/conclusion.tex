\section{Conclusion}

Using the results from \cref{sec:results}, the two research questions introduced in \cref{sec:research_questions} will now be answered.

When looking at the extent to which Elixir is easier to use when writing a multi-agent system compared to JADE, the empiric metric chosen was the number of lines of code that were required.
The results in \cref{tab:loc} show that Elixir required 495 fewer lines of code compared to JADE\@.
This results in around a third reduction in lines of code compared to JADE\@.
This shows that the provided OTP behaviours in Elixir can be easily used for agent-based design in a multi-agent system.
Furthermore, it shows that a lack of a standardised message format such as FIPA ACL does not hinder the design of a multi-agent system in Elixir.
Therefore, the answer to this research question would be that it is easier to write a multi-agent system in Elixir compared to JADE\@.

Using the results in \cref{sec:cpu_usage,sec:memory,sec:time} we can answer to what extent a system written in Elixir performs better than a system written using JADE\@.
We can see that Elixir performs better in terms of memory usage and the total time taken for an experiment, but performs worse in CPU utilisation when compared to JADE\@.
The CPU utilisation result is surprising but seems to be due to the difference in runtime compared with JADE, as well as the fact that the experiments provide no downtime for the agents.
Arguably, the most important metric is the time taken for the experiments.
This is because, in a real-world system, agents that perform their tasks as quickly as possible would be preferred.
For this metric Elixir performed between 1 and 2 orders of magnitude better than the equivalent JADE system.
The answer to this research question would be that for the most part, a system written in Elixir performs better than a system written in JADE\@.

The answers to both of the research questions asked in this dissertation indicate that Elixir could be a viable alternative instead of using JADE for multi-agent systems.
This is due to it both requiring fewer lines of code to implement a multi-agent system, as well as performing significantly better than a system using JADE\@.

These results are similar to the results that \citeA{fedrecheski2016elixir} found when comparing Elixir to Java.
They found that Java used less CPU load but Elixir used fewer lines of code and less memory.
This shows that the results of this project are accurate when compared to other research that has been done between these two languages.

\subsection{Reflection}

While this project was able to answer the research questions that were set in the beginning, there were still some major shortcomings.
The largest shortcoming was the lack of complexity in the decision making of the individual agents in the system.

While the agents can buy and sell their goods, a lot of planned strategic decision making had to be cut due to time constraints.
The agents currently only seek to sell their goods for an immediate amount of profit, with no behaviour to plan for the future.
This can be seen with the amount of money that the Manufacturer agent makes.
Either the Manufacturer agent makes a nice profit or spends the entire experiment haemorrhaging funds.
Similarly, there is a lack of handling for exceptional circumstances such as if an order were to arrive late or not at all.
Implementing more complex behaviour might have narrowed the performance gap between Elixir and JADE as Elixir is not a language where the performance of `number-crunching' tasks is prioritised, whereas in Java these types of workloads are more common.

This leads to the second shortcoming which is a lack of detailed performance analysis.
While the results measured show that JADE is significantly slower than Elixir, they don't show why JADE might be slower than Elixir.
This could be investigated by the use of detailed profiling.
This would allow us to see where the JADE system was spending most of its time, and whether this was something that could be fixed in the agent.

If performance issues were to be found in the JADE framework itself, it might be difficult for a fix to be applied.
As mentioned in the literature review, development on JADE has affectively halted since 2017.
During the writing of this dissertation, the original JADE website and project ended up being taken down.
Currently a fork of JADE is being maintained by C\'edric Herpson\footnote{\url{http://www-desir.lip6.fr/~herpsonc/en/}} at Sorbonne University, France.\footnote{\url{https://jade-project.gitlab.io/}}
It has yet to be seen how this will impact the use of JADE for multi-agent systems, but it highlights the importance of evaluating alternate technologies for research.

\subsection{Future Work}

Building off this dissertation, there are two avenues in which future work can be done.

The first would be to cooperate with other researchers in the field of multi-agent systems to determine what functionality they would require from Elixir.
This would help gauge the community interest in using Elixir for multi-agent systems.
It would also hopefully provide a list of features that Elixir would need to implement to be used in research.
As Elixir is a general-purpose programming language, these features would not likely be implemented at the language level.
However, Elixir supports the easy use of software libraries and has a centralised package repository.
This would allow any required functionality such as a FIPA compliant message protocol to be implemented as a library and then distributed as open-source software.

The other avenue would be to investigate other features of Elixir to see how they could be used when implementing multi-agent systems.
Currently, this dissertation mostly uses the features of OTP, built-in support for distributed systems and some of the support for concurrent processes.
However, a system could be designed that uses these features to their full extent as well as investigating how the language support for fault tolerance and hot-code reloading could be applied to multi-agent systems.
This system could also look at using the web-framework Phoenix\footnote{\url{https://www.phoenixframework.org/}} to provide a method of human-agent communication.
